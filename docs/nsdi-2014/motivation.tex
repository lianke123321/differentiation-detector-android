\section{Motivation and Goals}

%What is the problem we are targeting?
\meddle is motivated by the impractical nature of existing solutions that can diagnose the behavior of mobile devices and what ISPs do with the traffic naturally generated by these devices. 
Existing solutions rely on a combination of OS modifications~\cite{enck:taintdroid, hornyack:appfence, pathak:eprof, qian:mobilearo,wei:profiledroid}, static and dynamic analysis of mobile applications\cite{egele:pios, ravindranath:appinsight}, analysis of OS service logs\cite{falaki:smartphoneusage}, and traces collected by ISPs or at gateways~\cite{vallina-rodriguez:bfc}. \tbd{add more references from mobisys and imc paper}.
These solutions are impractical because they either violate the terms and conditions of device usage, they are heavily dependent on the specific version of the OS and application, they cannot scale to support new OSes, or they are limited to a specific geographic region.
%These restrictions  impractical nature thus creates a high-barrier to entry for with end-users.


We now define the goals for a practical user-friendly system for mobile diagnostics
\tbd{For each goal, What does it mean followed by why is it important? Needs rephrase}
\begin{packedenumerate}
\item \emph{Deployable.} 
Easy to install/use/configure. 
It must not require warrant voiding of the device. 
This is important to support a large user-base. 
\item \emph{On-demand}.
The user must be able to enable and disable service easily.
It must be easy to fallback to the original state.
This is important to ensure that users can easily opt-out and are not blocked if there are some problems with the system. 
\item \emph{Agnostic to OS, ISP, access technology, and applications.}.
The research work that comes out must not be limited to specific OS, ISP, or device manufacturer, or set of applications.
\item \emph{Scalable.} 
Can support many users. 
To ensure that research results have statistical significance. 
\item \emph{Always On.} 
Once enabled by the user, it must be passive and pervasive. 
It must not demand periodic inputs from end-users. 
%This is important to ensure passive and pervasive operations that do not demand periodic inputs from end-users.
%\item Flexible - supports many services
%\item Reliable - always works
%\item Safe - does not adversely affect traffic (interaction between modules)
%\item Secure - handles user data with care
\end{packedenumerate}


Indeed, there exists a trade-off between a practical user-friendly solution and a solution that offers a fine-grained control over mobile devices. 
Unlike existing approaches, we accept the bait of coming up with a practical and user-friendly solution and testing the limits of its usefulness. 
Specifically, we relinquish OS-level controls to focus on the Internet traffic generated by mobile devices and try to use this perspective to diagnose mobile devices and the ISPs that serve these devices. 

The network perspective is promising because mobile devices are increasingly becoming the primary gateway to access Internet based services. 
This vision is backed by the recent International Telecommunication Union (ITU) report: \emph{``in developing countries, mobile-broadband services cost considerably less than fixed-broadband services''}~\cite{ict:facts}.
The increasing Internet usage from mobile devices makes the traffic generated by these devices comparable to the traffic generated by PCs~\cite{falaki:smartphoneusage}.
This perspective becomes even more important because a large number of free applications use Internet based advertisements to make-up for their costs~\cite{pathak:eprof,vallina-rodriguez:bfc}.

Intuitively, a practical way to obtain this network perspective is to redirect all the mobile Internet traffic through a middlebox configured for traffic diagnosis. 
Offloading traffic diagnosis to a middlebox brings with it a potential to be agnostic to the OS, device manufacturer, application installed, and access technology used by the device. 
Such diagnosis through traffic redirection is practical only if the devices whose traffic we wish to diagnose support traffic redirection. 
Mobile devices natively support redirection using VPN tunnels and redirection through HTTP proxies.
This is done to support their enterprise clients. 

%%% Local Variables: 
%%% mode: latex
%%% TeX-master: "meddle-main"
%%% End: 
