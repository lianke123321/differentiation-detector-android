\section{Diagnosing Misbehaving Applications}
\label{sec:diagnose-apps}

The closed nature of mobile OSes hides the network behavior of mobile applications from the users of mobile devices. 
The results from previous studies on misbehaving apps is also not available on the stores from which applications are bought.
Thus users have to make their decisions based on the publicly available information on an apps behavior, \eg via reading reviews.
Furthermore, once an app is installed, users have limited options to diagnose the behavior of mobile applications and filter out misbehaving components~\cite{hornyack:appfence}. 

\begin{table*}[t]
    \centering
    \begin{small}
    \begin{tabular}{|l|l|l|l|l|l|l|l|l|l|}
       \hline
       {\bf Store}&{\bf Platform}&{\bf \# Apps}&{\bf Email}& {\bf Location}& {\bf Name} &{\bf Password}& {\bf Device ID}& {\bf Contacts}& {\bf IMEI}\\
       \hline
       App Store&iPhone&209&13 (6.2\%) &20 (9.5\%)&4 (1.9\%)&6 (2.87\%)&4 (1.9\%)&0 (0\%)&0 (0\%)\\
       \hline
       Google Play&Android&100&3 (3\%)&10 (10\%)&2 (2\%)&1 (1\%)&21 (21\%)&0 (0\%)&13 (13\%)\\
       \hline
%       Third Party&Android&908&1 (0.1\%)&32 (3.5\%)&2 (0.2\%)&0 (0\%)&95 (10.4\%)&4 (0.4\%)&48 (5.3\%)\\
%       \hline
    \end{tabular}
    \end{small}
    \caption{Summary of personally identifiable information leaked in plaintext (HTTP) by Android and iPhone apps. \emph{The popular iOS apps tend to leak the location information in the clear while Android apps leak the IMEI number and Android ID in the clear.}}
    \label{tab:pii}
\end{table*}

\tbd{We need to say nothing is novel in the results, but novelty lies on the ease of reproducing it even by end-users}. 

With the help of controlled experiments we now show how we used \meddle to identify misbehavior and to block traffic causing this misbehavior. 



%%% Local Variables: 
%%% mode: latex
%%% TeX-master: "meddle-main"
%%% End: 
