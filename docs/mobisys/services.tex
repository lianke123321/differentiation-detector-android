\section{Controlling Mobile Traffic}
In this section, we use two case studies to highlight some of the 
advantages \meddle provides as a new point of control 
for mobile networking traffic. We then discuss several new applications 
we are currently building.

\subsection{Packet Filtering}
Packet filtering is a common technique implemented in 
today's middleboxes. These can be used for implementing 
security policies, censoring content or preventing applications 
from harming the network (\eg, P2P). In \meddle, we use 
packet filtering to enable a number of features that are 
currently either unavailable or poorly implemented.

\noindent\textbf{Device-wide ad blocking.} Content and service providers 
typically use revenue from advertisements to cover their costs, enabling 
``free'' access for users. Given the additional costs in terms of data volume, 
battery consumption and potentially reduced privacy from tracking, it is unclear 
just how ``free'' these services truly are. Vallina-Rodriguez~\etal~\cite{Vallina-rodriguez:2012:AdCache} observe
that ads account for 5\% of daily traffic from more than 50\% of
Android users in a large European ISP. 

There is no current standard for opting 
out of unsolicited advertising; those that exist for tracking are not widely supported. 
Not surprisingly, many users have turned to software that blocks ad and analytic servers. 
For example, more than 20\,million users have installed 
AdBlockPlus~\footnote{http://www.adblockplus.org}, one of several tools for this purpose. 
Unlike the desktop environment, mobile device browsers do not provide support for 
ad blocking; further, most interactions occur inside of apps where browser plugins 
cannot help. 

\meddle makes it easy to implement an efficient, device-wide ad blocker. 
In our current implementation, we use a DNS-based filter to
block ads, analytics, and mediation sites.\footnote{The service is disabled by default, so users must 
opt in to enable blocking.} A key feature of our solutions is that it works 
regardless of whether SSL is used because DNS requests occur out of band 
from the secure connection. Further, the response from the DNS request is {\tt localhost}, 
meaning that devices will generate no external network traffic when failing 
to resolve the ad servers. 

\tbd{Ashwin: Insert graph for ad traffic.}

Our ad blocking engine relies on a publicly available list of domains for ads and
analytics~\cite{YoyoAds}; we augment this list of domains using 
recent research on mobile ads~\cite{hornyack:appfence,
  Leontiadis:2012:AdsMobile}. From our initial deployment, we observed a 0.05\% to 0.8\% reduction
in total traffic at each mobile device due to our ad blocking engine. 
\tbd{Ashwin: is this still true?}
In addition to the DNS-based filter, we are currently implementing a filter that blocks 
requests for URLs matching a regular expression, as done by AdBlockPlus.

\subsection{Traffic Manipulation}
Web tripnet stuff.

\subsection{New Applications}

Example Apps (highlight things we can do with meddle)
 - Begin with a sentence/intro-para on using middleboxes to offloading activities and offer device wide services
 - Middlebox based packet monitoring 
      - cross * possible
      - passive - real traffic 24x7 
      - actual users
      - Network traffic characterization
         - Longitudinal study of network traffic
         - understand behavior of apps
 - Device wide services - service like packet filtering that is not limited to an app 
     - Ad blocking
     - Platform for malware detection and blocking
     - Parental controls 
 - Deployment of new protocols and services 
   - Users can opt in for specific service
   - Mobile story for services like FreeDOM, CCNs, etc.
   - service in a VM where users opt in for services
 - Generic Proxy
   - Privad
   - Anti-censorship