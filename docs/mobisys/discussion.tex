\section{Discussion}
\label{sec:discussion}
In this section, we discuss several key issues and limitations for an indirection-based 
deployment such as \meddle.
 
\noindent\textbf{Privacy and trust.} As discussed in \S\ref{subsec:currdeploy}, we 
collect traces from users as part of an IRB-approved study. We take great care in 
protecting user privacy -- data is encrypted before being stored, the private key is 
not stored on the server where data is recorded and any PII sent in the 
clear by apps is stripped from our datasets as soon as we identify it. 

While this fine-grained data is useful for informing the design of meddlebox solutions (\eg, 
identifying and stripping personally identifiable information from unencrypted 
HTTP traffic), it can be prohibitive for a large-scale study. In the next phase of 
our deployment (currently under IRB review) we will capture only packet headers 
and lengths. With a lower privacy risk, we believe we can recruit a larger number 
of users and obtain informed consent via a Web site. 

Regardless, users may still be uncomfortable with sending all their traffic through a 
\meddle service, be it in a hosting center or in the cloud. We are making all of 
our code open source so that users can install \meddle on servers in their 
own networks (\eg, in their home network). Users may opt to use this option instead; however, they will also be responsible 
for updating \meddle to include the latest new features and bug fixes. We expect 
that most users interested in running \meddle will be content with using our hosted 
\meddle service with anonymized packet headers being collected. 

\noindent\textbf{Acceptable use.} Similar to any service providing Internet access, 
\meddle needs an acceptable use policy (AUP) to ensure that we are not liable for 
user activity. We model our AUP after the one provided by EC2, one of our potential 
hosting providers. Users are informed of this AUP at install time. If we are notified 
of an AUP violation, we can isolate the user generating the traffic because each 
user is given separate certificate-based credentials. This makes it easy to 
remove offending users without disrupting compliant users.

\noindent\textbf{ISP objections.} Many mobile carriers deploy in-network middleboxes 
for traffic engineering purposes. By tunneling \emph{all} traffic, these boxes lose the 
ability to implement ISPs' policies, which could lead to suboptimal performance for 
our users and for other users sharing the network. 

We believe that addressing such concerns is straightforward. First, we do not 
expect \emph{every} device to run \meddle. If we were to attract even 1\% of 
mobile users that would be a surprisingly huge success. Thus we do not expect 
\meddle to significantly impact overall traffic in a mobile network. Second, if \meddle 
traffic were to become a significant traffic engineering challenge, we argue that 
transparency can go a long way toward addressing the issue. Namely, 
given sufficient information we can implement policies to address problematic traffic using either 
meddleboxes or pigeonholes that send traffic outside of the VPN 
tunnel. 

\noindent\textbf{Scalability and reliability.} If successful, \meddle will 
face scalability and reliability challenges as the number of users increases. 
We believe that, in return for valuable measurement and experimentation data, 
we can justify funding to support some number of thousands of users. For a 
larger deployment, we may have to consider a paid service or alternative 
economic models. By running the service in the cloud and using DNS-based 
redirection, we expect service unavailability to be relatively rare. 

\noindent\textbf{Limitations.} We believe that \meddle enables a large number 
of interesting studies and experiments in the mobile environment. However, 
there are limits to its effectiveness. 

First, \meddle captures only network traffic. It does 
not allow us to gain access to sensors on the device, information about what 
apps are running or control over those apps on the device. An interesting question 
is what are the limits of what one can infer from a device's network traffic alone. 
We have shown that iOS network traffic exhibits clear, identifiable differences 
when plugged in versus running on battery, lending evidence that we can successfully 
infer properties of the phone previously available only by querying from an app 
on the device. Likewise, we used SSL certificates to distinguish which app generates 
requests to a CDN.

Second, \meddle may miss some data traffic. For example, we found evidence that 
iOS push was using signaling on a circuit-switched channel (SMS). We believe 
the volume of ``missing'' traffic is small; however, it remains to be seen how 
this holds generally and over time. As mobile networks transition to all-IP, it will 
be interesting to see if an approach like \meddle can indeed capture all of a device's traffic.

Last, \meddle incurs costs that may dissuade adoption. We discussed several of the 
overheads based on establishing a VPN tunnel. One area of future work is to improve 
VPN protocols to reduce the time to establish a connection. 
