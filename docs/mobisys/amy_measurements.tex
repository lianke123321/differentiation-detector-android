
    Smartphone users are often faced with the decision whether or not to trust an app downloaded from the iPhone App Store or the Android Market.
    The user can access some information \emph{a priori} regarding the app's behavior, \eg{} via reading reviews or inspecting the permissions requested, but this provides at best an incomplete and at worst and incorrect view.
    Although some undesirable app behavior is obvious -- \eg{} pushing advertisements to the notifications bar, installing unwanted third-party software, \etc{} -- the app's network behavior remains generally invisible to the user even after installation. \meddle allows users and researchers to see this behavior and draw conclusions about whether it is appropriate. 
     To guide our analysis in the rest of this section, we define several key network-usage properties that describe a ``well behaved'' app:
    \begin{itemize}
        \item Protection of personally identifiable information (PII): transmitting PII only when required for application functionality, and if transmitting PII, doing so over HTTPS rather than HTTP.
        \item Conservation of power and data quota by making only necessary use of network.
        \item Communication only with servers needed for the applications intended purpose, with preferably a limited number of entities contacted by the app overall.
    \end{itemize}

\noindent\textbf{Survey of privacy violations in the wild.} In our dataset, we were surprised to find several examples of PII sent in the clear. Specifically, we identified IMEI (device unique identifiers) and 
GPS coordinates in HTTP request strings. Offending apps include The New York Times, The Weather Channel, Fox News, ESPNCricInfo, FirstPost and 
Angry Birds. For many of these apps, the privacy violations are due to included software for displaying ads. The key takeaway is that privacy 
violations from tracking software is pervasive and often easy to detect. When using \meddle, users limit their exposure to eavesdroppers 
because the connection to the \meddle server is encrypted. Further, in the next section we discuss how \meddle can strip this information. 
  
\noindent\textbf{Controlled experiments.} While the above examples indicate clear examples of undesirable app behavior, 
it is more difficult to determine which traffic is strictly necessary for app functionality and which can be removed through blocking 
at a \meddle server. 
    With this in mind, we used \meddle to perform a small app-by-app analysis of network usage behavior.
    
       We downloaded six of the most popular apps from the Android Market and installed them on an fresh install of Android 4.1.
       The apps fall into the following categories: a mainstream image-sharing app, two popular games, a weather app, a settings 
        management app, and a risqu\'{e} app serving various images. We interacted with each app for between 7 and 15 minutes 
        to understand network behavior when the app is being used.

    \noindent{\bf Leaking PII.}
    None of the apps we installed for these experiments leaked PII over HTTP. However, there was evidence that the weather app was making use of a tracking service, contacting an third-party service which markets itself as specializing in ``device tracking and fraud prevention.''
    It is unclear why a weather app requires a service for fraud prevention.

    \noindent{\bf Third Party Contacts.} 
    The settings app and the game apps contacted a limited number of third parties.
    The settings app contacted Amazon, Google, a hosting service, and two CDN services.
    One game app contacted only Amazon, Google, and a single CDN provider, the other contacted only Amazon, Google, a private server belonging to the game developer, and three CDN/Ad services.
    The weather app, the image app, and the risqu\'{e} app both contacted a large number of third party services; the weather app contacted 10 and the risqu\'{e} app contacted 19. 

   \noindent{\bf Resource Consumption.}
   Finally, we considered resource consumption towards both battery life and data caps.

    The risqu\'{e} app and the image app by far exhibited the largest traffic volumes: on average, the risqu\'{e} app consumed 23.37 Kbps and the image app consumed 22.84 Kbps.
    Both apps were downloading large numbers of images.
    The weather app and settings app consumed 4.49 Kbps and 6.27 Kbps respectively; the games apps had much more limited bandwidth consumption with only 1.5 and 1.7 Kbps consumed each.

    To evaluate behavior towards battery life, we considered the fraction of 30-second intervals during which no traffic was sent by the phone, rather than the amount of traffic sent overall. 
    This metric captures cases where apps generate frequent traffic that causes the radio to stay powered for an extended period of time.
    The games apps, which were the best performers in terms of bandwidth consumption, were some of the worst performers in terms of radio usage: 78\% and 66\% of 30-second intervals generated network traffic.
    The worst performers overall were the settings app and the image app, which both generated traffic during 85\% of 30-second intervals.
    The weather app, and the risqu\'{e} app generated traffic during 23\% and 49\% of 30-second intervals respectively.

    \noindent{\bf Summary.}
    The risqu\'{e} app acted maliciously in installing unwanted apps; whether or not the weather app acted maliciously through its use of a third party tracking service is unclear.
    However, all apps, even those that were decidedly non-malicious, had some questionable network behavior by contacting multiple third parties and making inefficient use of bandwidth and battery life.
    \meddle can help ameliorate some of these concerns: data consumption and third-party contacts can be restricted by blocking DNS lookups and connections to ad services. 
    In \S\ref{sec:filtering}, we discuss these filtering capabilities further.

