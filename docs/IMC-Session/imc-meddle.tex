\documentclass[10pt]{article}
\usepackage{fullpage}
\usepackage{color}
\usepackage{xspace}

\newcommand{\meddle}{{Meddle}\xspace}
\newcommand{\tbd}[1]{[[[{\color{red}{\bf{TBD: #1}}}]]]}


\begin{document}



\title{\bf Meddle: Controlling and Characterizing Mobile Network Traffic}

\author{David Choffnes, Justine Sherry, Ashwin Rao, Arnaud Legout,\\
  Arvind Krishnamurthy, and Walid Dabbous.}

\date{}
\maketitle

\thispagestyle{empty}

We present \meddle, a platform that provides control and transparency
for mobile network traffic without requiring any operating system
modification. This platform can improve visibility into mobile networks
and enables a wide range of new experiments in this space. We would
like to get feedback on \meddle from researchers attending IMC and
recruit them to participate in an IRB approved measurement study.

\meddle is motivated by the following characteristics of today's
mobile systems. Currently users have little control of how their
devices use the mobile networks they pay for. In the mobile
environment, users are forced to interact with a single operating
system tied to their device, generally use closed-source apps provided
for the OS that routinely violate user
privacy~\cite{hornyack:appfence}, and subscribe to network providers
that can (and do) transparently modify, block or otherwise interfere
with network traffic~\cite{wang:middleboxes}.

Researchers face a similar set of challenges for characterizing an
experiment for mobile systems. To characterize mobile traffic and
design new protocols and services that are better tailored to the
mobile environment, we would like a framework that allows us to
intercept and potentially modify traffic generated by mobile devices
as they move with users, regardless of the device, OS or
carrier. However, implementing this functionality is difficult on
mobile devices because it requires warranty-voiding techniques such as
jail breaking to access and manipulate traffic at the network
layer~\cite{enck:taintdroid}. Even when using such an approach,
carriers may manipulate traffic once it leaves the mobile
device~\cite{wang:middleboxes}, thus rendering some research
impractical. Last, some protocols and services should be implemented
in the network instead of the device (e.g., prefetching and security
filters) but researchers generally have no ability to deploy such
solutions.
 
In this presentation, we will show that we can provide the necessary
framework to simultaneously address these issues for users and
researchers by using middleboxes accessible through VPN tunnels, an
approach we call \meddle. Once packets arrive at VPN servers, we can
use a variety of middlebox approaches to transform traffic to and from
mobile devices. This enables new research in both measuring and 
characterizing mobile traffic, and designing new in-network features
to improve the mobile experience. In addition to avoiding transparent
interference from any middleboxes inside mobile carrier networks,
\meddle enables researchers to investigate what-if scenarios for the
impact of new middleboxes as if they were deployed in carrier
networks.
  
To obtain a large number of participating users for characterization
and experimental evaluation, we need to explicitly align the goals of
researchers and users. \meddle offers sufficient
incentives for users to adopt the service by offering device-wide
ad-blocking, privacy/security filters and parental controls at the
network layer -- functionality that mobile network providers do not
currently make available. 

We are currently using our prototype \meddle architecture to build out
a system for public deployment. We are also recruiting users for an
IRB-approved study. 


\bibliographystyle{acm}
\begin{scriptsize}
\bibliography{hotnets-meddle-middle}
\end{scriptsize}
\end{document}

\end{document}

