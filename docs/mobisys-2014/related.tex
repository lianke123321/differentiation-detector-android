\section{Related Work}
\label{sec:related}

The network behavior of mobile systems has implications for battery life, 
data-plan consumption, privacy, security and performance, among others. 
When attempting to characterize this behavior, researchers face a number 
of trade-offs: compromising network coverage (limiting the number and type of ISPs measured), 
portability (limiting the device OSes) and/or deployability (limiting subscriber coverage).
\platname compromises 
none of these, enabling visibility and control of network traffic across carriers, devices and access 
technologies. Table~\ref{tab:relatedCompare} puts our approach in context with related  
approaches regarding network behavior of mobile systems. 

\begin{table*}[t]
\begin{center}
{\footnotesize
\begin{tabular}{|l|l|l|l|l|l|}
\hline
 & \textbf{Network Coverage} &  \textbf{Portability} &  \textbf{Deployment model} &   \textbf{Meas. Type}  & \textbf{Control?} \\ \hline
AT\&T/Telefonica study~\cite{vallina-rod:ads,gerber:passivespeed} & Single carrier & All OSes & Instrument cell infrastructure & Passive & No \\ \hline
WiFi study~\cite{chen:wifi} & Single WiFi network & All OSes & Instrument WiFi network & Passive & No \\ \hline
PhoneLab~\cite{PhoneLab}/TaintDroid~\cite{enck:taintdroid} & Multiple networks & Android & Install custom OS & Active/Passive & Yes \\ \hline
MobiPerf~\cite{wang:middleboxes}/SpeedTest~\cite{sommers:cellwifi} & Multiple networks & Android & Install App & Active & Yes \\ \hline \hline
\platname & Any network & Most OSes & VPN configuration & Active/Passive & Yes \\ \hline
\end{tabular} }
\end{center}
\vspace{\postfigspace}
\caption{Comparison of related work. \platname is the first approach to provide visibility and control over network traffic for all access 
networks and most device OSes.}
\label{tab:relatedCompare}
\vspace{\postfigspace}
\end{table*}%

Traces from mobile devices can inform a number of interesting analyses. Previous work 
uses custom OSes to investigate how devices waste energy~\cite{pathak:eprof}, network bandwidth and 
leak private information~\cite{enck:taintdroid,hornyack:appfence}. Similarly, AppInsight~\cite{ravindranath:appinsight} and PiOS~\cite{egele:pios} can inform 
app performance through binary instrumentation and/or static analysis. In this work, we explore the opportunity to use network traces 
alone to reveal these cases without requiring any OS or app modifications. 

Network traces from inside carrier networks provide a detailed view for large numbers 
of subscribers. For example, Vallina-Rodriguez~\etal~\cite{vallina-rod:ads} use this approach to characterize performance and 
the impact of advertising. Gerber \etal~\cite{gerber:passivespeed} similarly use this approach to 
estimate network performance for mobile devices.  %\cite{maier:mobtraffic} \cite{chen:wifi}
Similar to these approaches, \platname provides continuous passive monitoring of mobile network 
traffic; however, \platname is the first to do so across all networks to which a device connects.

Active measurements~\cite{wang:middleboxes,sommers:cellwifi} allow researchers to understand network topologies and instantaneous 
performance at the cost of additional, synthetic traffic for probing. In contrast, \platname uses 
passive measurements to characterize the traffic that devices
naturally generate. PhoneLab~\cite{PhoneLab} provides a mobile experimentation platform with 
low-level OS and device access; however, it does not run on unmodified device operating systems. 
In contrast, \meddle does not require OS modification---facilitating large-scale, global deployment.



%From DJW @ CCS:
%more privacy leakage shown from zero-permission Android APIs
%(Identity, Location, Disease and more ...). there are API calls
%needing no permissions to find out the identity of running  apps, the
%amount of data they send over the network, the BSSID (through ARP
%data), when an app is using the speaker, etc. combined with external
%data this can lead to user identification. example: a local app can
%see when you tweet given the characteristic lengths of tweeting, and
%can correlate that with public twitter data to identify who you are.
%to complete attacks they also send data with zero permission by using
%the browser and encoding the data to be sent as part of the request.
%SPARTA folk: maybe more APIs to add to the suspicious list, or other
%attacks to consider?

% AppIntent, a tool to relate the GUI context to data transmission,
%based on the assumption that transmissions resulting from a GUI
%interaction are intended/authorized by the user. this is on my reading
%list, and likely relevant to SPARTA folk and Ravi.

%nice study by Wagner folks on how android apps misuse crypto, i.e.,
%use it but don't get the security they wanted. many apps encrypt data
%with the standard APIs, but in dumb ways: the default engine is ECB
%mode, which wasn't what you wanted, and IVs are often set to
%constants, etc. they do a program analysis to find this stuff and
%observe that close to 90% of apps get it wrong. yikes! this sort of
%stuff should just go away with decent tools. SPARTA folk: most of
%these analyses look amenable to information flow. plus another
%analysis would be to assure that all data that is sent went via the
%encryption APIs.

%
%-"format-transforming encryption", a twist on format preserving
%encryption, neatly mapped one protocol to look like it had the format
%of another. the motivation was to get new protocols through
%middleboxes by making them look like HTTPS (say). techniques like this
%plus the rise of TLS will surely complicate DPI. i wonder how much
%longer DPI has or how it will evolve?

%%% Local Variables: 
%%% mode: latex
%%% TeX-master: "main.tex"
%%% End: 
