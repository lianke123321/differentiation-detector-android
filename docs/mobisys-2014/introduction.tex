\section{Introduction}
\label{sec:introduction}

%what is the problem area you are working in and why is it important? It is important to set the larger context here. Why is the problem of interest and importance to the larger community?

Mobile systems consist of walled gardens inside gated communities, i.e., locked-down operating systems running on devices that interact over a closed and opaque mobile network. 
Despite a large collection of privacy, policy and performance issues in mobile networks~\cite{enck:taintdroid,hornyack:appfence,speedtest,ma:edoctor,pathak:eprof,bickford:mobilemalware}, researchers are faced with few options to characterize and address them.

\drc{Tell a good story here. Our work explores the opportunities for improving visibility and control in mobile networks when presented only with a network flow. We demonstrate this simple interface in enormously powerful -- it allows us to investigate and block privacy leakage, detect and avoid ISP interference/service differentiation and identify and remediate malicious third-party attacks -- all without modifying apps, the OS or infrastructure. }

%What is the specific problem considered in this paper? This paragraph narrows down the topic area of the paper. In the first paragraph you have established general context and importance. Here you establish specific context and background.
%Cannot mention user participation and crowd sourcing here because we have not achieved it in this work . 
%What we have here is a platform with a potential for user participation and preliminary results. 

%The key challenge is that mobile OSes and ISPs provide no built-in service to diagnose the network traffic generated by these applications. 
%\tbd{Why is this important? Why do we care?}.
%As a result, previous studies~\cite{vallina-rod:ads,gerber:passivespeed,chen:wifi,enck:taintdroid,wang:middleboxes,sommers:cellwifi} are constrained by at least one of the following: mobile OSes, access technology, device manufacturer, installed applications, and user behavior.
%In this work, we are the first to present an approach that compromises none of these, an approach that can be used across carriers, mobile devices, apps, and access technologies.
%
%\meddle redirects all Internet traffic through a software-defined middlebox for the purpose of analysis and interposition.
%Specifically, \meddle builds on the native support for VPN tunnels and HTTP proxies by mobile OSes to tunnel all the Internet traffic regardless of the access technology used by the device. 
%A shortcoming of this approach is that it compromises the fine grained view offered by existing solutions in favor of being user-friendly.
%
%In this paper, we use \meddle to test the limits to which traffic redirection can be used to diagnose mobile Internet traffic. \tbd{too weak also mobile Internet traffic needs to be rephrased.}. 
%The main contributions of this paper are as follows:
%\begin{packedenumerate}
%\item Platform for mobile diagnosis through network traffic analysis and control. Single server solution empowers users to install and configure them on home-gateways.
%Researchers can deploy them for measurement studies. 
%\item Controlled experiments using off-the-shelf Android and iOS devices. 
%\item Controlled experiments to analyze ISP interference in US and France. 
%\end{packedenumerate}
%
%The remainder of this paper is structured as follows.

%%% Local Variables: 
%%% mode: latex
%%% TeX-master: "meddle-main"
%%% End: 

