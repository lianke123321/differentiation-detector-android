\section{Application: \\
Revealing ISP Behavior}

This section describes how we build two applications atop \meddle to 
reveal the policies ISPs apply to mobile traffic traversing their networks. 
Previous work focused on addressing this problem in fixed-line networks; to 
the best of our knowledge we are the first to provide this functionality for 
mobile systems.

\subsection{Web Tripnets: Detecting ISP Content Manipulation}

ISPs, middleboxes and client software are known to change Web page content for 
a variety of reasons including performance optimization and security. In some cases, 
a third party can change a page for selfish reasons, \eg to insert ads that generate revenue 
for that party. Figure~\ref{fig:tripnet-example} depicts an example of content injection in China, where 
a banner ad is replaced by information about the local airport.

\begin{figure}
\centering
\includegraphics[width=0.9\linewidth]{figures/injectioncrop.png}
\caption{Screen capture of content injection by a Chinese ISP in November, 2013. The 
highlighted region at the bottom should be an advertisement from a US company. }
\label{fig:tripnet-example}
\end{figure}


This problem of Web interference was first highlighted by Reis~\etal~\cite{reis:tripwires}. 
The authors demonstrated that although a small percent of users were affected by in-flight changes, those changes tend to introduce vulnerabilities including cross-site scripting (XSS) attacks~\cite{reis:tripwires}. 
They proposed and deployed \emph{Web Tripwires}, a Javascript code to detect in-flight page changes. 
The main limitation of \emph{Web Tripwires} is that it requires each Web site to modify their content to include a tripwire.

In \meddle, we extended tripwires to alleviate this limitation. 
Namely, we use the HTTP proxy present in \meddle to inject a tripwire on \emph{any} page 
without requiring support from Web site developers -- an approach we call a \emph{Web Tripnet}.

\noindent\textbf{Implementation.} With \meddle all traffic is tunneled, thereby 
preventing ISPs from modifying pages. To help inform non-users of ISP content 
manipulation, we provide the Tripnet as an opt-in feature. Because this entails 
two fetches of every Web page, we also support two modes: always-on and low-rate random trials, where we 
insert tripwires for some small fraction of their visited sites.

The Tripnet works as follows (Fig.~\ref{fig:tripnet}. A client requests a Web page through the \meddle VPN 
tunnel. This request is forwarded to the destination server. The response returns to 
the \meddle server, where a transparent proxy injects the tripwire code.\footnote{We 
recognize the irony of injecting content to detect content injection, but this is done only with user consent.}
The tripwire-enabled response is forwarded to the client, which execute the javascript at page load time.


\begin{figure}
\centering
\includegraphics[width=0.9\linewidth]{figures/tripnet.pdf}
\caption{Overview of the \meddle Web-tripnet. A Web page is loaded
through the VPN tunnel, where a meddlebox inserts a Web tripwire. The
tripwire causes the browser to reload the Web page using the address of a
transparent proxy server that is accessed using an unencrypted connection.
After the proxied version of the page is loaded, the browser or a meddlebox
can compare the two pages to identify potential interference. }
\label{fig:tripnet}
\end{figure}

The tripwire code contains information about the page content prior to traversing the ISP. When 
executed, the code fetches the page again to compare with the (known) unmodified page content.  
To ensure that this fetch does \emph{not} traverse the VPN connection, we use a \emph{pigeonhole} domain 
whose traffic traverses an untunneled interface. For example, if the original request was for www.facebook.com, 
we send the request to tripnet.meddle.mobi/www.facebook.com/, where we run a Web proxy. 

When the request arrives at our proxy server, we could forward the request to the original target. In 
practice, however, doing so would return different content due to the highly dynamic nature of most 
Web content. Instead, we cache Web pages at the tripnet-injecting server and co-locate our Web proxy 
there. Thus we return exactly the same Web page that was received over the tunneled connection. 
Any difference in page content can only be due to ISP behavior. 


\noindent\textbf{Sites and ISPs tested.} We conducted controlled experiments using our Tripnet 
architecture, using the top 100 Web sites according to Alexa. We tested using 
AT\&T, T-Mobile and Verizon in the US, and Orange, Sosh and Bouygues in France. 
Many sites customize content according to User-Agent strings, so we spoof User-Agents 
as coming from iOS, Andriod and desktop clients.

\noindent\textbf{Page modification/injection.} Cases of manipulation of text content.

\noindent\textbf{Content modification.} In addition to detecting changes to text content in 
Web pages, we also use our controlled experiments to investigate whether ISPs are manipulating 
media content, \eg downsampling high-resolution images to reduce bandwidth consumption 
from mobile devices. For this experiment, we augment our Tripnet experiments with the result 
of wget results from the mobile device and the proxy server. Note that this experiment requires an 
app or tethered laptop to collect the Web media objects fetched over the mobile network.

Our results are as follows.

\noindent\textbf{Implications.} 


\subsection{App-Agnostic Replay: Detecting Service Differentiation}

Overview of the approach and challenges. \meddle allows us to capture 
traces from mobile devices over both \wifi and cell, meaning we can model 
behavior from either medium. \meddle also serves as a convenient location 
to conduct replay experiments. The challenge is how to capture and replay 
the salient features of application traffic such that it will be subject to differentiation 
from middleboxes. State what we can reproduce (timings, sequence of bytes, ports and source IPs) 
and what we cannot (destination IPs from client). Challenges with UDP. Challenges 
with different network conditions. Differences from Glasnost.

\textbf{Graph}: If we find a case of difference between \wifi and cell, plot it here.

Assumptions: ISPs will differentiate traffic based on hostname, IP addresses, ports, total number of 
connections, payload signatures, total bandwidth, time of day, ... \tbd{anything else?}

\noindent\textbf{Approach and feasibility.}
Description of adopted replay approach and tests to show validity.

\textbf{Graph}: Similarity score for network traffic in setting where we know there is no differentiation. 
Score can include overall throughput, latency characteristics, loss, packet timings.

\noindent\textbf{Wide-area testing.} Networks tested, observed behavior.

\textbf{Table}: each row is an app, each column is an ISP, each cell is a marker indicating what kind of 
differentiation is happening

\textbf{Graph}: Plot showing a sequence of traces that demonstrates what this differentiation looks like



\noindent\textbf{Content modification.} Downsampling, e.g.

\noindent\textbf{Implications.} 