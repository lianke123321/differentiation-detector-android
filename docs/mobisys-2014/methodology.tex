\section{Datasets Description}
\label{sec:dataset}
Using \platname, we collected full packet traces from Internet activity generated by
mobile devices. We use this data to study how to map monitored traffic to applications, and to
analyze PII leakage. Below, we describe our data-collection methodology, which consists of
1) controlled experiments in a lab setting and 2) IRB-approved ``in the wild'' measurements 
gathered from real users during seven months.

\subsection{Controlled Experiments}
\label{sec:dataset-contr-exper}
Our goal with controlled experiments is 1) to obtain ground truth information 
about network flows generated by apps and devices, and 2) characterize the 
network activity for a large variety of popular apps in a lab setting. We use 
this data to understand how to map network flows to the app that generated 
them, and how to identify PII in those network flows. 

\noindent\textbf{Device setup.} We conducted our controlled experiments using three devices: a Galaxy
Nexus running Android 4.2, a Google Nexus running Android 4.0, and
an iPhone 3GS running iOS 6. We start each set of controlled experiments
 with a factory reset of the device to ensure that software installed by previous 
 experiments cannot impact the network traffic generated by each device. 
 Then we connect the device to the
\platname{} platform, we enable the SSL-Bumping plugin, and begin
the experiment. 

\noindent\textbf{Manual tests.} We manually test the
100 most popular free Android apps in the \emph{Google Play} store and 209
iOS applications from the iOS App store on April 4, 2013. For each
application, we install it, enter user credentials for the account if
it is relevant, interact with it for up to 10 minutes, and uninstall
it. This allows us to characterize real user interactions with popular applications 
in a perfectly controlled environment. Note that 
because we enter a unique and distinguishable set of user credentials when 
interacting with apps, we can easily extract the corresponding PII from 
network flows (if they are not obfuscated).

\noindent\textbf{Automated tests.} The second set of controlled experiments consist of fully-automated
experiments on the most popular 908 Android applications from a free,
third-party Android market, \emph{AppsApk.com}~\cite{appsapk}.
We perform this test because Android devices can install
\emph{Third-party applications} that are not available on the
\emph{Google Play} store, without requiring the user to root the device. 

Our goal is to understand how these apps differ from those in the standard \emph{Google Play} 
store, as they are not subject to Google Play restrictions.
%\tbd{Is there different constraints on this free market, AR: They do not have paid application. All apps must be free.}
We automate experiments using \emph{adb} to
install each app, connect the device to the \platname{} platform, and
start the app. Then we use \emph{Monkey}~\cite{adbmonkey}, an app-scripting 
tool, to perform a series of 10,000 actions that include
random swipes, touches, and text entries.  Finally, we use adb to
uninstall the application and reboot the device to forcibly end any
lingering connections. This set of experiments is limited to
Android devices because iOS does not provide equivalent 
scripting functionality. 

% The results of our controlled experiments can be found in
% \fref{sec:manual-testing}.

\subsection{In The Wild Measurements}
\label{sec:dataset-wild-measurements}

The controlled experiments in the previous section provide us with 
ground-truth information for a large number of apps running in a controlled 
setting for a short period of time. To understand the network behavior of 
devices with real users "in the wild" over longer time periods, we conducted 
an IRB-approved measurement study with a small set of subjects, from 
Oct. 20, 2012 to May 20, 2013.\footnote{The measurement study is ongoing, we report the most recent subset of results.}

We deployed two \platname servers, one in the USA and one in France
that were used by 26 devices: 10 iPhones, 4 iPads, 1 iPodTouch, and 11
Android phones.  The Android devices in this dataset include the
Nexus, Sony, Samsung, and Gsmart brands while the iPhone devices
include one iPhone~3GS, four iPhone~5, and five iPhone~4S.  These
devices belongs to 21 different users, volunteers for our IRB approved
study.  This dataset, called \mobWild, consists of 208 days with data; the number of 
days for each user varies from 5 to 205 with a median of 35 days.  For privacy reasons, the
SSL-Bumping plugin is \emph{disabled} for all measurements involving
real users.

Capturing all of a subject's Internet traffic raises significant
privacy concerns.  Our IRB-approved study entails informed consent
from subjects who are interviewed in our lab, where the risks and
benefits of our study are clearly explained.  The incentive to use
VPNs is Amazon.com gift certificates awarded by lottery. To protect the
identity of information leaked in the data, we use public key
cryptography to encrypt all data before storing them 
on disk; the private key is
maintained on separate secure severs and with access limited to
approved researchers.  Further, subjects are free to delete their
data and disable monitoring at any time.  Per the terms of our IRB, we cannot 
make this data publicly available due to privacy concerns. We are investigating 
alternative data-collection techniques that provide user anonymity sufficient 
for sharing with other researchers.

\subsection{Descriptive Statistics from User Study}

This section highlights key features of the dataset gathered from 
users participating in our study (\mobWild). 
Due to the relatively small number of users in our study, we cannot 
draw strong, generalizable conclusions; rather, we use data gathered 
from them to demonstrate that there is a need for \platname{} to 
obtain a comprehensive view of Internet traffic from mobile devices. 

\noindent\textbf{Network connectivity.} First, we describe the diversity 
of networks that our users connected to. 
We infer the access technology (WiFi or cellular) using the AS description from \emph{WHOIS} data for each IP address used by a mobile device.
Based on this classification, the \mobWild dataset consists of traffic from 54 distinct ASes, of which 9 are cellular ASes.
During the measurement study, each device connected to our \platname server from at most two distinct cellular ASes. 
In contrast, a median of 4 \wifi ASes were observed per device and for one device we observed traffic from 25 different \wifi ASes spread across 5 countries.
In terms of traffic volumes, collectively our users' devices transferred 24-56\% of their traffic over cellular (iOS and Andriod, respectively), and the 
remainder over WiFi. 
The key take-away is that, \emph{for our users}, instrumenting a single cellular carrier or WiFi access point misses a 
large fraction of traffic generated by mobile devices. \platname{} avoids this limitation.

\noindent\textbf{Traffic protocol classification.} We begin our identification process using the classification provided Bro~\cite{bro}.
Bro uses the protocol field in the IP header to broadly classify the flows, and we use this classification to label flows as either TCP, UDP, or \emph{other}.
Bro further classifies TCP flows using well defined port numbers, and we use this classification to label flows as either HTTP, SSL (which includes HTTPS, IMAP, etc.) or \emph{other} flows.
Similarly, we use Bro to label UDP flows as either DNS or \emph{other}. 
Table~\ref{tab:summaryIOSAndroidTraffic} presents a 
summary of traffic generated by user devices in our study. 

There are two key  
take-aways from this table. First, Web and SSL traffic dominate traffic for users in \mobWild, 
and there is significant diversity in the usage patterns for 
users with Android and iOS devices. For example, more than 92\% of the traffic in our \mobWild dataset is either HTTP or SSL, 
the fraction of total flows over cellular or \wifi differ significantly for each OS. This motivates the need for a platform that covers 
multiple OSes and multiple access technologies. Second, a significant portion of flows occur over secure (SSL) connections 
that generally prevent classification using deep packet inspection. This calls into question 
the overall effectiveness of traffic optimization approaches that rely 
on middlebox technologies that interpose of plaintext traffic (\eg page rewriting or downsampling media).

\begin{table}
\begin{small}
\begin{center}
\begin{tabular}{|p{0.11\columnwidth}|p{0.14\columnwidth}|r|r|r|r|}
\hline
{\bf IP} & \multirow{2}{*}{\bf Service} & \multicolumn{2}{|c|}{\bf Android} & \multicolumn{2}{|c|}{\bf iOS} \tabularnewline
\cline{3-6}
{\bf Protocol} &           &  \textbf{Cell.}  &  \textbf{\wifi}  &  \textbf{Cell.}  &  \textbf{\wifi}  \tabularnewline
\hline
\multirow{3}{*}{TCP}
       &  HTTP (\%)  & 35.39 & 68.67 & 52.11 & 75.51 \tabularnewline
\cline{2-6}
       &  SSL (\%)   & 61.14 & 27.37 & 46.76 & 18.76 \tabularnewline
\cline{2-6}
       &  other (\%) & 2.34  & 3.29  & 0.26  & 1.81 \tabularnewline
\hline
\multirow{2}{*}{UDP}
       &  DNS (\%)   & 0.67  & 0.49  & 0.56  & 0.34  \tabularnewline
\cline{2-6}
       &  other (\%) & 0.32  & 0.11  & 0.27  & 3.57  \tabularnewline
\hline
 Other &  other (\%) & 0.14  & 0.07 & 0.04  & 0.01  \tabularnewline
\hline
\multicolumn{2}{|c|}{\emph{total (\%)}} & 100.00 & 100.00 & 100.00 & 100.00 \tabularnewline
\hline
\multicolumn{2}{|c|}{\emph{Traffic Volume (GB)}}& 9.467 & 18.79 & 14.89  & 89.54 \tabularnewline
\hline
\multicolumn{2}{|c|}{\emph{\# Flows}}   & 851074 & 658331 & 661736 & 2167448 \tabularnewline
\hline
%\multicolumn{2}{|c|}{\emph{\# Devices}} & 10 & 11 & 10 & 15 \tabularnewline
%\hline
\end{tabular}
\end{center}
\end{small}
\caption{Traffic volume (in percentage) of popular protocols and services on Android and iOS devices over cellular and \wifi.
\emph{TCP flows are responsible for more than 90\% of traffic volume. Traffic share of SSL over cellular networks is more than twice the traffic share of SSL over \wifi.}} 
\label{tab:summaryIOSAndroidTraffic}
\end{table}


An important question for network characterization is which app is responsible for which 
network flows. As we demonstrate in the following section, previous approaches are insufficient 
for mapping the majority of apps to their corresponding network flows. We describe 
several techniques to improve this mapping, and present results for controlled experiments 
and the \mobWild dataset.



%%% Local Variables: 
%%% mode: latex
%%% TeX-master: "main"
%%% End: 


