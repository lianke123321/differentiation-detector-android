\section{Application: Mobile Malware \\
Detection and Blocking}
\label{sec:malware}

%\subsection{Mobile infection vectors}
%
%\textbf{Two-second approach}: Get list of drive-by downloads for mobile. Potential sources:
%- Websites that post this stuff
%- Tweets by fake accounts from Alan's twitter data\
%- Other?
%
%Make a script that causes a device to load those malicious URLs. Identify malicious behavior from there.
%
%\textbf{Table}: Types of malicious URLs and what they do. Rows are URL types, columns are types of behavior, cells are counts/percentages/checkmarks. 

%\subsection{Mobile malware network analysis}

In the fixed-line environment, malware is used for botnets, spam, data exfiltration, and sabotage, among other 
purposes. Despite a handful of studies in the mobile space, little research has focused on the \emph{network 
properties} of malware in the mobile environment. In this section, we use \meddle to investigate malware network 
behavior and develop new techniques for detecting and blocking malware activity based exclusively on network 
activity.

\mypara{Assumptions and methodology} We base our analysis on the likely incentives for malware authors, mainly 
that they distribute malware for profit. In this setting, we observe that malware \emph{must} communicate some information 
to a remote service for coordination (\eg C\&C), exfiltration (stealing personal information) or billing (contacting toll services). 
We can detect the first two cases if they use IP; we cannot detect circuit switched activity such as sending SMS or making 
phone calls.

\meddle gives us two opportunities to detect and block malware activity over IP. First, we can detect 
the app binary being downloaded via a hash and block that transfer if it is identified as malware. Second, we can use network 
behavioral analysis to identify malicious activity from apps that passed through our hash filters and successfully installed. 
In the following paragraphs we describe how existing hash registries fail to capture most confirmed malware, how to 
use network activity to identify and block malware and how to augment hash registries by automatically identifying the 
malicious app binary responsible for the activity -- thus inoculating other users from this malware. Importantly, 
this approach works at machine instead of human timescales.%, protecting infected users and preventing new infections 
%within minutes, not days.

\mypara{Dataset} To understand malware network behavior, we use a 
dataset consisting of 111 confirmed malicious Android APKs gathered by the Andrubis project~\cite{andrubis} in September, 2013.  
Similar to the approach in \S\ref{sec:dataset-contr-exper}, we conduct controlled experiments by installing 
a malware app on a clean OS with dummy user data, interacting with it for \tbd{X} minutes and recording the 
network traces using \meddle. To compare with non-malicous apps, we use the dataset gathered in \S\ref{sec:dataset}.

The malware consists of 46 families ranging from backdoors to spyware.  99 (89\%) of the apps generated network 
traffic, of which 91\% generated HTTP traffic. Many of the remaining apps targeted earlier versions of the Android 
OS and thus did not work in our tests. Note that although our study focuses on Android malware, our approach 
is applicable to iOS and other mobile OSes because it relies only on the ability to monitor network traffic.

\mypara{Signature-based detection is insufficient} First, we determine whether existing malware hash registries 
contain signatures for the malicious APKs in our data. Even in MONTH (X months after the malware was identified 
by Andrubis), we find that only Y\% of the apps in our dataset were correctly identified. This motivates the need 
for behavioral analysis.

\mypara{Malware exhibits distinctive network behavior} 
Previous work in fixed-line environments~\cite{perdisci:malwaresig} use network behavioral analysis to identify malware 
from network flows. In particular, they rely on \useragent information and perform clustering on this to distinguish 
malware from non-malicious traffic. In our dataset, we find that ...

Of the 99 apps generating network traffic, we observe that 81\% (73)  
exfiltrate PII. Unique identifiers such as IMEI, Android ID and MAC address are most commonly leaked, and 
we further find 9 apps leaking telephone numbers. For the remaining 26 apps, we did not detect PII;12 use HTTP  
traffic, 7 use nonstandard ports and 1 attempts to communicate with a server that is not online. 

\mypara{Effectiveness of behavioral filters}
TBD 

%\subsection{Unsolicited traffic}
%Presumably there is none, but probably worth looking at inbound unsolicited traffic from a 
%\meddle instance with no users compared to one with active users. Might find something interesting there.
%
