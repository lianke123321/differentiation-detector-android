\section{Feasibility}
\label{sec:eval}
The previous section provided several examples of new research opportunities that \meddle enables. In this section, we highlight several key questions regarding the feasibility of deploying our approach at scale.

\subsection{Overhead}
\label{subsec:cost}
Because our approach in part depends on users installing a VPN configuration and tunneling all traffic through \meddle, we evaluate whether the cost to the user in terms of performance, power and data quota is sufficiently low.

\noindent\textbf{Power consumption.} Tunneling traffic to a \meddle
server requires that all traffic be encrypted by the mobile
device. While this is already commonly performed for SSL connections,
\meddle requires an \emph{additional} layer of encryption. We observed
a 10\% increase in power consumption when streaming an HD video to
Android and iPhone devices using our IPsec tunnel. We believe that this 
overhead is reasonably low, and we note that this cost for encryption comes with the added benefit of increased 
privacy from carriers.

An interesting research question is whether it is possible to \emph{reduce} power consumption using \meddle. For example, Qian et al~\cite{qian:rrc,qian:aro,qian:periodic} found that traffic shaping (a service that \meddle provides) can significantly reduce the power consumed by devices when periodic application traffic and radio resource timers are out of sync.

\tbdal{Is it possible to assess the power consumption based on the
  same experiment as Data consumption? This way we can factorize the
  explanation. }
\noindent\textbf{Data consumption.} IPsec encapsulation slightly inflates packet sizes, in addition to
preventing carrier middleboxes from applying their own compression. We
measured the overhead of the tunnel in terms of data overhead from IPsec 
headers and keepalive
messages, finding that it ranges from 8--12\%. For our measurements we setup \meddle
as a VPN gateway for an iPhone and Android phone. On each phone
we accessed the Internet using the VPN tunnel for about one hour. Our
test traffic was generated by activities that we expect to be typical of mobile 
device usage: Web searches, map searches, online shopping, 
downloading popular apps, emailing and reading the news. We also uploaded a picture to 
Facebook and Twitter, streamed a video on YouTube, and played a popular game (Angry Birds). 
%In terms of dollars, this corresponds to YYY. Keep-alive messages?

\noindent\textbf{Performance.} By forcing user traffic to an
intermediate server and interposing on flows, we may add latency both
due to additional hops and due to processing time at the \meddle
server. We envision a DONAR-style deployment where users are
dynamically redirected to different \meddle servers based on network
conditions and server load~\cite{wendell:donar}. Given this model, we
evaluate whether we can locate servers near mobile-network egress
points using a deployment such as PlanetLab, and found that this is
generally the case.

For this experiment, 
we used data from approximately 10 mobile phones located throughout
the US and issued traceroutes from the devices to targets in Google
and Facebook's networks. We then used the first non-private IP address seen 
from the mobile device on the path to a server. We assume that this corresponds 
to the first router adjacent to the mobile carrier's public Internet egress point. Note that we could not simply ping the device IPs because mobile carriers filter inbound ping requests. Using this set of egress adjacencies, we determined the round-trip time from each PlanetLab site, then took the average of the nearest five sites to represent the case where a host at the nearest site is unavailable due to load or other issues. The average latency to each router was between 3\,ms and 13\,ms, with a median of 5\,ms. Thus, when compared to RTTs of 10s or 100s of milliseconds that exist in mobile networks, the additional latencies from traversing \meddle servers is expected to be relatively small or even negligible.

\tbd{What is the cost in terms of end-to-end performance?}

\tbd{Are there cases where we actually improve performance due to
detouring? (I've decided not to mention this because it's not a sure thing.)}
\tbdal{In case the carrier is performing traffic differentiation to
  penalize some bandwidth consuming traffic, using a VPN might
  significantly improve performance. I don't know whether traffic
  differentiation is something common on mobile networks. (that was an
  answer to Dave question. If he discards it, also drop my comment.) }

\subsection{Deployability}
In this section, we discuss the challenges for deploying \meddle at scale. 

\noindent\textbf{Portability.} To be successful, \meddle should be
supported by nearly all mobile devices and be easy to deploy on
servers. In our current implementation, \meddle uses native IPsec to
establish VPN tunnels and the Vyatta middlebox software to shape
traffic. Both of these software artifacts are supported on vanilla
Linux operating systems, which in turn run on nearly all servers. For
mobile devices, Android, BlackBerry and iOS systems all support VPNs;
Windows phones are expected to add support in version 8 of the
OS. Manually installing a VPN generally requires filling out five
fields on a Android phone, and the VPN configuration can be
distributed using a single file on iOS. We have tested that our server
software runs correctly on Amazon's EC2 offering and are currently
working on adding kernel modules to support it on PlanetLab and Vicci
nodes.

\noindent\textbf{Scalability.} If wildly successful, we would like to
ensure that \meddle scales gracefully and that there are sufficient
resources to support large numbers of concurrent users
worldwide. Based on our initial analysis using StrongSwan on commodity
hardware, we found that each connection consumed on average less than
1\% of CPU time. Thus, we expect to be able to support up to 100s or
small number of thousands of users per server, which is in line with
low-end VPN appliances sold by Cisco and Vyatta. A recent
study~\cite{pcworld-speedtests} showed that current rates for 3G
networks in the US were between 0.59 and 3.84 Mbps; assuming devices
are uniformly distributed across carriers, we expect to be able to
support 250 simultaneous users (saturating their download capacity)
for every 1 Gbps of bandwidth at the server.

% DRC: Removing this per Justine's suggestion because it's not a technical problem.
%\noindent\textbf{Cost.} The need for potentially hundreds of instances of \meddle running simultaneously begs the question of how will we pay for this. Using current EC2 pricing, we estimate that running a meddle deployment that supports 1000 users, each user generating 1\,GB of traffic per month, will cost \$30/mo in CPU and storage cost and \$115/mo in bandwidth cost -- less than 15 cents per user. For PlanetLab, there is no additional cost, but the trade-off is performance. We envision a number of solutions. First, several large corporations have a vested interest in the performance and freedom that \meddle 
%provides for mobile Internet traffic and it is possible to obtain
%sponsorship from them. In another funding model, users pay for
%differentiated services, thus subsidizing the \meddle deployment for
%free users. Last, we note that we intend to release open-source 
%code for \meddle, allowing users to run their own \meddle servers
%in their home networks. While this would reduce researchers' ability
%to experiment with traffic and would likely reduce performance for the
%user (if the home network is far from the mobile network egress
%point), this is nonetheless a viable option for users who want the
%features of \meddle without needing to trust our infrastructure or pay
%a premium for superior service.
%\tbdal{Do we plan to release opensource code and configuration for
%  meddle. It might be worth mentioning if this is the case. }


\section{Discussion}
\label{sec:discuss}

In this section, we discuss several open questions, limitations of our approach and how to address them.

\noindent\textbf{User incentives.} We claim that users would like more control over how their mobile devices interact with 
the network and that the incentives for \meddle adoption are sufficient to attract a diverse and large user population. One 
could argue that users do not need or want more control over the network, and that non-experts would find it hard to take 
advantage of the opportunities that \meddle affords. In response to this objection, we note that large numbers of users and policymakers 
were outraged by recent studies about information leaks from apps~\cite{wsj:apps-watching-you}. Further, there is a large set of users installing ad-blocking software in Web browsers -- 
if tens of millions of people do it, it likely falls in the domain of non-experts. 

We do not expect any single incentive for user adoption to be sufficient. Rather, the 
research enabled by \meddle should form a positive feedback loop in which new, proven research artifacts become additional 
incentives for user adoption, thus enabling further research.

\noindent\textbf{Trust and Privacy.} \meddle eliminates the need to trust carriers or devices with network traffic; however, 
this comes at the cost of users needing to trust \meddle servers. We currently use state-of-the-art IPsec implementation where a user and a \meddle system authenticate themselves to each other using digital certificates.

Beyond the trust concerns, \meddle provides a tap on network traffic that could allow researchers to see all 
unencrypted packets generated by devices -- a serious risk for violating user privacy. We must ensure that 
researchers are restricted to capturing only summary information about packets (e.g., headers and packet length), 
that user identifiers are removed from any persistently stored dataset
and that users explicitly give informed consent 
for any tracing, experimental features or other forms of traffic manipulation. To help instill confidence from users, 
\meddle software will be implemented as open source artifacts made publicly available, and users will have 
the option to run their own instance of \meddle (with their own root of trust) if they so desire.

\noindent\textbf{Limitations.} While \meddle offers several new opportunities for research, there are limitations 
to the control it offers. First, \meddle currently cannot control what apps are installed on phones or the network 
traffic they generate and thus cannot address many network problems that occur ``at the source." Second, although 
\meddle can modify packet timings to play nicely with protocols and timers in the mobile environment, it cannot 
be used to unilaterally effect new protocols between the mobile device and \meddle server. Last, \meddle has limited ability 
to impose DPI-based policies and protocols when a connection to a server is encrypted (e.g., via SSL). It is possible 
that users can install a root of trust enabling \meddle servers to proxy these sessions, but this 
raises additional privacy concerns.

\noindent\textbf{Interference.} One of the key features of \meddle is that it takes control of network traffic 
away from mobile network providers. This is a half truth, because carriers can choose to 
block communication with \meddle servers. However, 
there are strong disincentives for this behavior because it not only violates net neutrality, but also is likely to pose a public relations nightmare for a carrier accused of doing so. In 
a similar vein, advertisers and the app providers that they support are likely to find ways to avoid our filters 
for blocking ads, much like the ongoing cat-and-mouse game in desktop Web browsers. We 
do not claim to have a solution to end the game, but we argue that the continued effectiveness of ad blocking 
in desktop scenarios is likely to persist in the mobile environment.


%%% Local Variables: 
%%% TeX-master: "hotnets-meddle-middle.tex"
%%% End:
