\section{Introduction}
\label{sec:intro}


Mobile networks are the most popular, fastest growing and least
understood systems in today's Internet ecosystem. Despite a large
collection of privacy, policy and performance issues in mobile
networks~\cite{enck:taintdroid,hornyack:appfence,speedtest,eprof}, users and researchers are faced with few options to
characterize and address them. 
%%Justine: I adore this next sentence.
The crux of the problem is 
that mobile systems consist of walled gardens inside gated 
communities, i.e., locked-down operating systems running 
devices that interact over a closed and opaque mobile network. 

%Arnaud: commented out and replaced with the text below in
%\ty. Discussing bandwidth limitation does not bring much in that
%context. 
%
% From the user perspective, the problem is that subscribers are paying
% for access to the Internet much like they do for home wired service,
% but in the mobile environment they receive relatively little bandwidth
% and control in return.  
From the user perspective, the problem is that subscribers to
  mobile networks have little control compared to what they have in
  their wired access networks.
% DRC: Agreed and changed the text.
%\tbdal{I am not sure the argument of money is relevant and
%  appropriate. In many countries (latin america countries, spain,
%  etc.) in which the wired phone network is under-developed, wireless
%  access is cheaper and more efficient than wired access.}
For example, in the home network, users are accustomed to the ability
to install custom applications that change application network usage
(e.g., ad blocking) and to run home routers that implement policies
such as network access prioritization and parental controls. In the
mobile environment, however, users are forced to interact with a
single operating system tied to their device, generally use
closed-source apps provided for the OS that routinely violate user
privacy~\cite{hornyack:appfence}, and subscribe to network providers
that can (and do) transparently modify, block or otherwise interfere
with network traffic~\cite{wang:middleboxes}.  Users thus need a
mechanism to take back control of how their devices use the mobile
networks they pay for.

The lockdown of mobile devices places researchers in a similar bind.
To characterize mobile traffic and design new protocols and
services that are better tailored to the mobile environment, we would like a
framework that allows us to intercept and potentially modify traffic
generated by mobile devices as they move with users, regardless of the
device, OS or carrier. However, implementing this functionality is
difficult on mobile devices because it requires warranty-voiding
techniques such as jail breaking to access and manipulate traffic at
the network layer~\cite{enck:taintdroid}. Even when using such an
approach, carriers may manipulate traffic once it leaves the mobile
device~\cite{wang:middleboxes}, thus rendering some research
impractical. Last, some protocols and services should be implemented
in the network instead of the device (e.g., prefetching and security
filters) but researchers generally have no ability to deploy such
solutions.

%For example, studies such as those from TaintDroid researchers 
%and The New York Times highlighted the kinds of personally 
%identifiable information being leaked to app servers and third 
%parties from popular apps. The XXX study showed that mobile 
%carriers interfere with traffic by blocking ports and/or surreptitiously 
%responding to traffic on behalf of third party servers. Finally, 
%3GTest and SpeedTest studies provide insight into mobile 
%network performance but only at the time when a user initiates 
%a measurements. 

In this paper, we argue that we can provide the necessary framework to
simultaneously address these issues for users and researchers by using
middleboxes accessible through VPN tunnels, an approach we call
\meddle.  \meddle works for nearly all mobile devices out of the box:
Android, iOS, and Blackberry all support VPNs, thus providing a
portable mechanism to tunnel traffic to servers outside of the
carriers' control regardless of the mobile device's network.  Once
packets arrive at VPN servers, we can use a variety of middlebox
approaches to transform traffic to and from mobile devices. This
enables new research in both measuring and characterizing mobile
traffic, and designing new in-network features to improve the mobile
experience.

In addition to avoiding transparent interference from any  
middleboxes inside mobile carrier networks, \meddle 
enables researchers to investigate what-if scenarios for the 
impact of new middleboxes as if they were deployed in carrier 
networks. In \S~\ref{sec:apps}, we discuss several research 
directions enabled by this service including new app-accelerators, 
mobile-specific security filters and protocol manipulation 
to improve power consumption and data volume usage. Importantly, 
service providers and users can take advantage of these features 
without requiring any support from carriers or new OS-specific 
apps installed by users.

%Arnaud: I remove this content, I am fine with the new text
%\tbd{Arnaud had a comment about device-wide services. I'm not 
%sure how to incorporate this now.}

To obtain a large number of participating users for characterization 
and experimental evaluation, we need 
to explicitly align the goals of researchers and users. We argue that 
\meddle offers sufficient incentives for users to adopt the service by 
offering device-wide ad-blocking, privacy/security filters 
and parental controls at the network layer -- functionality that mobile 
network providers do not currently make available.  
\tbd{JS: can we state that this functionality is /not/ available on today's devices? And how would parental controls work given that the kid can just turn off the VPN?}
\tbd{This is covered in sec 3 -- do you think it belongs here?}
% DRC: Agreed.
%\tbdal{I am not convinced that content compression is a striking
%  argument here. There are two reasons to compress content: reduce
%  the data plan usage, improve page loading. It is not clear that we can really compress data (may
%be by a few percent on average), but it is most likely that pages will
%load slower unless with have an overprovisionned VPN
%infrastructure. Content compression sounds like Opera mini that is
%targeted to old devices, so not a very exiting feature with some
%competitors. As you write later, it might be a research exploration,
%but certainly not an incentive at the beginning of the service.
%In section 3.1 you do not consider content compression, but security
%instead. Security is a much stronger argument.}


While the technologies that enable \meddle are well understood, there 
are a variety of challenges and open questions we must address to ensure a solution 
that is practical both for end users and researchers. Importantly, we 
must design a VPN-server deployment that does more good than harm 
with respect to performance and power consumption so that we avoid a disincentive for usage (\S\ref{sec:eval}). In addition, we 
would like to investigate the extent to which we can enact optimizations that 
reduce page load times, improve security, reduce data consumption and potentially 
even conserve power -- all from a middlebox that resides neither on the 
device nor in the carrier's network. Finally, we must address the additional 
security, trust and privacy concerns that arise when tunneling traffic outside 
of carrier networks into a third-party distributed service. We discuss these issues and others 
in Section~\ref{sec:discuss}.


%%% Local Variables: 
%%% TeX-master: "hotnets-meddle-middle.tex"
%%% End:
