\section{The Case for a New Platform}
While the goals of \mobilelab are noble, the literature is littered with Internet measurement platform 
proposals. Why should our proposal succeed? In this section, we argue that the 
mobile environment poses unique constraints that make traditional, ad-hoc measurement techniques impractical. 

We begin by considering several existing successful approaches for providing 
Internet measurements such as PlanetLab, M-Lab, traceroute servers and BGP feeds. 
These vantage points have facilitated 
a wide array of fundamental research including understanding routing performance, deploying and measuring 
large-scale distributed systems such as distributed hash tables and content distribution networks, and 
studying and improving transparency for backbone and broadband performance. We believe the 
key to their success is that they expose a simple interface for issuing measurements, 
require little-to-no software or infrastructure management from the researcher, 
and/or make it easy to access measurement data. 

We would like to replicate these 
keys to success in our platform, but before addressing this challenge we ask 
whether we can simply leverage these existing systems. Unfortunately, the answer is no; 
the following key features of mobile networks prevent existing measurement 
systems from meeting the goals of our work.

\noindent\textbf{Visiblity.} Mobile networks are currently designed in a way that roughly resembles a 
large private network residing behind a firewall/NAT that is connected 
directly to a tier-1 ISP~\cite{wang:middleboxes}. As a result, most of the infrastructure in 
mobile networks is unreachable from external vantage points.  

\noindent\textbf{Location.} Even if we were to deploy monitoring 
infrastructure inside  mobile networks, the question remains 
where to place it. Most of the interesting behavior occurs 
at or near the edge of mobile networks, making an infrastructure-based 
deployment costly. For example, Xu et al. point out that 
monitors located near a cellular provider's core cannot capture 
sufficient detail to account for detailed user mobility, and 
adding the infrastructure to support these measurements would 
require at least two orders of magnitude larger deployment~\cite{xu:acculoc}. 

\noindent\textbf{Rapid evolution.} The Internet, for better or worse, has 
largely ossified around a small core set of protocols and 
technologies, meaning that many aspects of measurement deployments for  
topology mapping, route monitoring and performance modeling 
can rely on little change to the underlying infrastructure 
and protocols. For example, TCP has been relatively unchanged since 1981, 
BGPv4 has been used since 1994 and IPv6 has made relatively little 
progress since 1998. In contrast, mobile networks are constantly 
evolving as new 3GPP specs are adopted (7 release versions since 2000), meaning protocols and 
infrastructure can be displaced within a few years. As a 
result it is difficult to recruit and retain representative and diverse 
measurement hosts in the mobile environment, and the resulting 
cost of infrastructure-based deployments can be prohibitively 
high. 

\noindent\textbf{Scarce resources.} Traditional network measurement 
platforms (rightfully) assume that the marginal cost of resources such as 
bandwidth and power are essentially zero. In mobile networks, 
unsolicited probes can significantly drain device batteries~\cite{wang:middleboxes} 
and bandwidth is several orders of more costly than in the 
wired setting. This makes concurrent and uncoordinated measurements fundamentally unscalable. 
For example, consider a small portion of the measurement traffic issued by a PlanetLab host: 1\,GB of 
ping traffic per day~\cite{nakao:underlay}. When run on a smartphone, this workload would drain 
battery power and consume monthly cellular data-volume quota within a fraction of a day\footnote{AT\&T's 250\,MB monthly plan for \$30/month.}
Even a single transfer-rate test using a 50 MB file 
can cost as much as $\$6$ when considered as a fraction of a user's monthly quota.
To put these numbers in context, the average app costs less than $\$2$
 and such as test would cost less 
than $\$0.02$ on a popular US broadband provider (Comcast).\footnote{\url{http://socialtimes.com/average-iphone-app-prices-rising-but-not-by-much_b70115}}

In addition to putting constraints on the amount of measurements we can perform, 
device management of scarce resources exacerbates interference between concurrent 
measurements. For example, mobile devices disable the radio during periods of 
inactivity, so a ping measurement that wakes the radio (and experiences additional 
delay during the wakeup period) can affect a subsequent ping measurement that 
occurs while the radio is in full power mode.

In the next section, we present a measurement platform architecture that explicitly 
accounts for these features of mobile networks while preserving the 
appealing properties of existing measurement systems.


%It turns out that the cost of 
%developing a new interface, managing the systems and collecting durable data is 
%prohibitively high. 
%Conducting research in this 
%environment is relatively straightforward because managing vantage-point software and hardware 
%is not the direct responsibility of participating researchers, so experiments are as simple as 
%issuing measurements and/or downloading results. Consequently, 
%
%
%First, devices in mobile networks are often subject to severe constraints in terms of power and 
%data volume quota, making concurrent and uncoordinated measurements fundamentally unscalable. 
%For example, consider a small portion of the measurement traffic issued by a PlanetLab host: 1\,GB of 
%ping traffic per day~\cite{nakao:underlay}. When run on a smartphone, this workload would drain 
%battery power and consume monthly cellular data-volume quota\footnote{AT\&T's 250\,MB monthly plan, for example.} within hours. As we show in \S~\ref{sec:constraints}, an open mobile 
%measurement platform can manage the effective use of a collection of devices through efficient measurement 
%scheduling and reuse.
%
%Second, in contrast to testbeds such as PlanetLab, it is difficult to recruit and retain representative and diverse 
%measurement hosts in the mobile environment particularly because of the rapid pace of change in mobile devices 
%and mobile network infrastructure. Researchers currently work with a small number of carriers and/or with small 
%deployments of apps, neither of which is sufficient to generalize results. A mobile measurement platform 
%can address the issue of managing measurement hosts, allowing researchers to focus on measurements instead of 
%management. In \S~\ref{sec:deployment}, we discuss a deployment and participation model that incentivizes 
%broad adoption of a mobile measurement platform.
%
%Infrastructure-based solutions are infeasible: AccuLoc says that detailed in-network 
%monitoring might require two orders of magnitude more hardware, and near the edge where 
%there are more physical locations to cover~\cite{xu:acculoc}.
%
%The primary motivation for this work is that research in the mobile environment is hampered by 
%the absence of a global, flexible and available platform on which to run measurement 
%experiments and the lack of an open repository for making the results of this data publicly 
%available. For example, previous work has relied on modified mobile OSes~\cite{} and/or 
%measurement-specific apps~\cite{}. Conducting this research required building/testing software to perform 
%measurements and record their results, acquiring and 
%managing a set of test devices, recruiting users to run the software on devices and finally gathering 
%data from these devices for subsequent analysis. A researcher new to the field is looking at several man-months or 
%years of labor to begin collecting data, not including the time required for IRB approval. Understandably, 
%such researchers are reluctant to release their data ``too soon''. Another model for previous work 
%is to rely on measurements from carriers~\cite{}. These results are often subject to nondisclosure agreements that 
%prevent other researchers from reproducing or conducting alternative analyses on the data. 

