\section{Network Characteristics of Operating System Services}
\label{sec:characterize-os}

Mobile operating systems provide APIs and OS level services to optimize network usage.
For example, iOS applications can use the Apple Push Notification Service (APNS) to receive notifications from the Internet. 
Similarly, iOS and Android APIs limit the background activity of applications to limit the network usage and extend the battery life of mobile devices. 
In this section we perform a set of controlled experiments to detail the network characteristics of the OS services.
The questions that we answer in this section are as follows.
\begin{packedenumerate}
\item How different is the network traffic from iOS devices compared to Android devices?
\item What are the network characteristics of operating system services?
\item What is the impact of operating system services in the wild? \tbd{rephrase this}
\end{packedenumerate}

\subsection{Network Characteristics of \tbd{Shipped/Factory Reset} Devices}

We now detail the network characteristics of devices that contain only the pre-installed applications.
We use the following questions as to guide our analysis
\begin{packedenumerate}
\item What is the network usage of devices that are used \emph{out of the box}? 
\item How does the device, manufacturer, and operating system affect the network usage?
\end{packedenumerate}

We began this experiment by first performing a factory reset on an iPod Touch, an iPad, an Samsung Galaxy SIII, and a Google Nexus S Phone.
All the devices had their batteries fully charged before the factory reset was performed. 
After the factory reset, we allowed these devices to connect to the Internet using our \wifi hotspot.
We use the same dummy email account on each device as the primary account for that device. 
We ran tcpdump on our hotspot to monitors the Internet traffic from these devices for 48 hours. 
For the first 24 hours each of the devices had a dummy gmail account as the primary device account.
For the final 24 hours we added a facebook account to the list of accounts.  
We use the data collected as a rough estimate on the minimum data traffic that is generated by the devices. 



\begin{figure}
\tbd{x axis is the hour, y axis the amount of data downloaded in that hour in a log scale. log scale to account for the first hour.}
\caption{Time evolution of traffic volume. \emph{We observe that in the first few minutes of connection establishment, the devices download the maximum amount of data.}} 
\label{fig:time-evol-factory-reset}
\end{figure}

In \fref{fig:time-evol-factory-reset} we observe that the devices exchange a significant amount of data during the first few minutes of the first boot. 
Despite the same account we observe that \tbd{the iPad} has the significantly larger traffic volume compared to the other devices. 
We also observe that after the first hour, the traffic volume is minimal. 
We observed that UDP corresponded to a maximum of \tbdv{number} of flows, and the \tbdv{x\%} UDP flows were DNS requests observed before TCP connections were established. 

\begin{table}
\begin{small}
\begin{tabular}{|p{0.20\columnwidth}|p{0.12\columnwidth}|p{0.12\columnwidth}|p{0.12\columnwidth}|p{0.18\columnwidth}|}
\hline
{\bf Device Type} & {\bf Total Volume} & {\bf TCP Volume} & {\bf TCP Flows} & {\bf Median Flows/Hour} \tabularnewline
\hline
iPod Touch & a & a & a & a \tabularnewline
\hline
iPad & a & a & a & a  \tabularnewline
\hline
iPhone & a & a & a & a \tabularnewline
\hline
Samsung Galaxy SIII & a & a & a & a \tabularnewline
\hline
Google Nexus S & a & a & a & a \tabularnewline
\hline
\end{tabular}
\end{small}
\caption{Traffic summary of devices after factory reset.}
\label{tab:factory-reset}
\end{table}

As show in \ref{tab:factory-reset} we observe that the median number of flows per hour is negligible. 
For the iOS devices we observe that the primary contributor for the traffic is the iOS push notification. 
We detail the characteristics in the rest of this section. 
During the 48 hour interval we observed that the \tbdv{device} produced the largest number of TCP and UDP flows \tbdv{number} of flows.

\tbd{Specific results come here. Motivate iOS push here for iOS devices.}
Placeholder for para on specific results.

\tbd{Discussion comes here}
Placeholder for discussion.

\subsection{iOS Push Notifications}

We now present the results of a detailed case study on the Apple Push Notification Service (APNS) that is used to push notification to iOS devices. 
We focus on APNS because it represents an OS-managed service on an operating system that received little research attention due to its closed-source nature. 
We use the following questions as to guide our analysis: 
\begin{packedenumerate}
\item What is the network impact of push notifications and does the measured activity coincide with published documentation? 
\item Are we able to capture the relevant data using traffic redirection?
\item What is the impact of traffic redirection using VPNs on the iOS push notifications? 
\item Can we use network traffic alone to infer the state of the device?
\end{packedenumerate}

The Apple Push Notification Service (APNS) implements push for iOS. 
The documentation for APNS provides limited details about the implementation, but does specify expected behavior (\eg, push
connections are established over cellular connections even if \wifi is available)

We explicitly verified all provided documentation and confirmed that all statements are true with the exception of the notification behavior with an iPad. 
The documentation states that the iPad will always remain associated with a \wifi AP, even if it is not plugged in. Our
experience shows this is not the case on an iPad 2.

\tbd{Ashwin: Text from Arnauds experiments}

\subsection{iOS Push in the Wild}

We now use the data from our \moball dataset to further investigate the behavior of iOS push over time for \tbdv{four characteristic users}. 
The objective of this analysis was to answer the following questions
\begin{packedenumerate}
\item How frequently do Push notifications take place in the wild?
\item What is the impact of access technology on push notifications?
\item What is the distribution of traffic volume of push notifications?	
\item \tbd{How efficient are services like Do Not Disturb?}
\item \tbd{How does push notifications change over OS upgrades?}
\end{packedenumerate}

\begin{figure}
\tbd{Ashwin: x axis is the hour of the day, y axis is the median number of push notifications received. plot for 4 users, 2 iphones and two 2 ipads}
\caption{Push notification in the wild (time of day).}
\label{fig:push-evolution-wild} 
\end{figure}

\begin{figure}
\tbd{Ashwin: x axis is the device id sorted by frequency of push notifications, y axis is the median number of push notifications received per hour. Data for 14 users that includes iOs device and iPhone, two points for each user: one for cellular and one for wifi}
\caption{Frequency of push notification in the wild.}
\label{fig:push-frequency-wild} 
\end{figure}

\begin{figure}
\tbd{Ashwin: x axis is the hour of the day, y axis is the number of push notifications received. plot for 4 users, 2 iphones and two 2 ipads}
\caption{Push notification in the wild (time of day).}
\label{fig:push-evolution-wild} 
\end{figure}









\subsection{Discussion}


