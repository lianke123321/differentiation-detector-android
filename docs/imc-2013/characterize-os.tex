\subsection{Network Characteristics of Operating Systems}
\label{sec:characterize-os}

Mobile operating systems provide APIs and OS level services to optimize network usage.
For example, iOS applications can use the Apple Push Notification Service (APNS) to receive notifications from the Internet. 
Similarly, iOS and Android APIs limit the background activity of applications to limit the network usage and extend the battery life of mobile devices. 
In this section we perform a set of controlled experiments to detail the network characteristics of the OS service.
The questions that we answer in this section are as follows.
\begin{packedenumerate}
\item How different is the network traffic from iOS devices compared to Android devices?
\item What are the network characteristics of operating system services?
\item What is the impact of operating system services in the wild? \tbd{rephrase this}
\end{packedenumerate}

\subsubsection{OSes In Factory Default}

We use \platname to detail the network characteristics of devices that contain only the preinstalled applications.
We use the following questions as to guide our analysis: 
\begin{packedenumerate}
\item What is the empirical network impact when devices are used \emph{out of the box}? 
\item How does the device, manufacturer, and operating system affect the network usage?
\end{packedenumerate}

We perform this experiment by first performing a factory reset on an iPod Touch, an iPad, an Samsung Android phone, and a nexus Android.
We then connect these devices to the Internet using our \wifi hotspot that monitors the Internet traffic from these devices for 12 hours. 
We use the data collected as the baseline for the minumum data traffic that is generated by the devices. 

\subsubsection{iOS Push Notifications}

We now present the results of a detailed case study on the Apple Push Notification Service (APNS) that is used to push notification to iOS devices. 
We focus on APNS because it represents an OS-managed service on an operating system that received little research attention due to its closed-source nature. 
We use the following questions as to guide our analysis: 
\begin{packedenumerate}
\item What is the empirical network impact of push notifications and does the measured activity coincide with published documentation? 
\item Are we able to capture the relevant data using traffic redirection?
%\item What is the impact of a VPN connection on network behavior? 
\item Can we use network traffic alone to infer the state of the device?
\end{packedenumerate}

The Apple Push Notification Service (APNS) implements push for iOS. 
The documentation for APNS provides limited details about the implementation, but does specify expected behavior (\eg, push
connections are established over cellular connections even if \wifi is available)

We explicitly verified all provided documentation and confirmed that all statements are true with the exception of the notification behavior with an iPad. 
The documentation states that the iPad will always remain associated with a \wifi AP, even if it is not plugged in. Our
experience shows this is not the case on an iPad 2.

\tbd{Ashwin: Text from Arnauds experiments}

\subsubsection{iOS Push in the Wild}

We now use the data from our \moball dataset to further investigate the behavior of iOS push over time for \tbdv{three characteristic users}. 
The objective of this analysis was to answer the following questions
\begin{packedenumerate}
\item How frequently do Push notifications take place in the wild?
\item What is the impact of access technology on push notifications?
\item What is the distribution of traffic volume of push notifications?	
\item \tbd{How efficient are services like Do Not Disturb?}
\item \tbd{How does push notifications change over OS upgrades?}
\end{packedenumerate}

\tbd{Ashwin: Text here based on the results}

\subsubsection{Discussion}

