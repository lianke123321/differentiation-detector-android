\section{Dataset Description}
\label{sec:dataset}

In this section, we describe the datasets we used in the subsequent sections.


\tbd{Run genDataSetDescription.R to get these numbers} 
The \moball dataset consists of mobile data traffic traces from \tbdv{25} devices that belong to \tbdv{19} users who are volunteers of an IRB approved study. 
This dataset consists of \tbdv{9} iPhones, \tbdv{4} iPads, \tbdv{1} iPodTouch, and \tbdv{11} of Android phones.
The Android devices in this dataset include the Nexus, Sony, Samsung, and Gsmart brands. 
The users of the 25 devices are spread across France and USA. 
This dataset consists of \tbdv{176} days of data that flowed through our VPN servers; the number days for each user varies from \tbdv{5} to \tbdv{176} with a median of \tbdv{32} days.  

\tbd{we need some wording and consitency for the usage of ISP -- for example ATT can provide cellular and DSL. Also mobile data cannot be used and we need some word for cellular data and wifi data and this must be defined in the dataset description.}
The \moball dataset consists of data traffic from \tbdv{} distinct ASes, of which \tbdv{} served cellular services (\fref{sec:abc} for details of AS classification).
Of the 19 devices that used cellular data, we observed that 16 devices restricted their cellular data traffic to one ISP each; the other three users used four, two, and two ISPs respectively. 
The number of \wifi ISPs per device was larger, the median number of ISPs observed was 4 with a maximum of 24 for one user. 
The user who contributed 24 distinct ISPs used \platname when traveling across 6 different countries. 
This implies \platname was able to capture traffic of users ``on the move.''
Campus wide studies such as \tbd{}, studies on DSL networks \cite{maier:mobtraffic}, and studies limited to traces from one specific ISP~\cite{vallina-rod:ads} are not able to capture this behavior.

%%% Local Variables: 
%%% mode: latex
%%% TeX-master: "main"
%%% End: 
