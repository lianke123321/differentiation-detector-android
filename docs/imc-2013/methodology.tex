\section{Datasets Description}
\label{sec:dataset}
Using \platname, we collected two different datasets that we use in the
next sections to analyze key characteristics of iOS and to compare them
with Android. In the following, we describe these two datasets, one
dataset has been collected using controlled experiments, the other
dataset has been collectected using IRB-approved in-the-wild
measurements during seven months on a small set of real users. 

The collection of these two very different datasets shows the
flexibility of \platname{} to perform a large variety of measurements
on the traffic of mobile devices. 

\subsection{Controlled Experiments}
\label{sec:dataset-contr-exper}
We made all our controlled experiments using three devices: one Galaxy
Nexus running Android 4.2, one Google Nexus running Android 4.0, and
one iPhone 3GS running iOS 6. We start each set of controlled experiments
 with a factory reset. Then we connect to device to the
\platname{} platform, we enable the SSL-Bumping plugin, and we start
the experiment. 

The first set of controlled experiments consist in manually testing the
100 most popular free Android apps in the \emph{Google Play} store and \tbd{}
iOS applications from the iOS App store \tbd{give a date}. For each
application, we install it, enter user credentials for the account if
it is relevant, play with it for \tbd{} minutes, and uninstall
it. This experiment is a characterization of popular applications with
real user interactions in a perfectly controlled environment. 

The second set of controlled experiments consist in fully-automated
experiments on the most popular 908 Android applications from a free,
third-party Android market\tbd{which market, we must give the name}.
We perform this test because Android devices can install
\emph{Third-party applications} that are not available on the
\emph{Google Play} store. So, it is important to characterize these
applications that do not have to follow the Android market publication
process\tbd{Is there different constraints on this free market}. To
automate the experiment process we use the \emph{adb} Android command shell to
install each app, connect the device to the \platname{} platform, and
start the app. Then we use \emph{Monkey} \tbd{give a ref}, an adb
stress tool, to perform a series of 10,000 actions which includes
random swipes, touches, and text entries.  Finally, we use adb to
uninstall the application and reboot the device to forcibly end any
lingering connections. This second set of experiment is limited to
Android devices because iOS does not provide an equivalent to adb to
manage apps installation. 

% The results of our controlled experiments can be found in
% \fref{sec:manual-testing}.

\subsection{In The Wild Measurements}
\label{sec:dataset-wild-measurements}

The controlled experiments described in
Section~\ref{sec:dataset-contr-exper} are important to characterize
the behavior of applications in a controlled environment. However, it
is also important to characterize the behavior of applications when
used by real users during several months. For this reason, we
performed an IRB-approved measurements during seven months, from 
October 20, 2012 to May 20, 2013 with a small set of real users.

We deployed two \platname servers, one in the USA and one in France
that were used by 26 devices: 10 iPhones, 4 iPads, 1 iPodTouch, and 11
Android phones.  The Android devices in this dataset include the
Nexus, Sony, Samsung, and Gsmart brands while the iPhone devices
include one iPhone~3GS, four iPhone~5, and five iPhone~4S.  These
devices belongs to 21 different users, volunteers for our IRB approved
study.  This dataset, called \mobWild, consists of 218 days of data
monitored on the \platname{} servers; the number of days for each user
varies from 5 to 215 with a median of 35 days.  We note that the
SSL-Bumping plugin has been disabled for all experiments involving
real users.

Capturing all of a subject's Internet traffic raises significant
privacy concerns.  Our IRB-approved study entails informed consent
from subjects who are interviewed in our lab, where the risks and
benefits of our study are clearly explained.  The incentive to use
VPNs was a lottery of Amazon.com gift certificates. To protect the
identity of information leaked in the data, we use public key
cryptography to encrypt all the tcpdump outputs; the private key is
maintained on separate secure severs and with access limited to
approved researchers.  Furthermore, users are free to delete their
data and disable monitoring at any time.  For privacy reasons, we will
not make this data publicly available.

%%% Local Variables: 
%%% mode: latex
%%% TeX-master: "main"
%%% End: 


