\section{Methodology and Dataset}
\label{sec:Methodology}

We used \platname to characterize mobile Internet traffic, and detail the impact of access technology and operating systems on application behavior. 

Our analysis methodology included controlled experiments to detail the behavior of specific applications and OS services, and a 7-month long IRB approved measurement study to characterize mobile Internet traffic in the \emph{in the wild} . 

\subsection{Controlled Experiments}

For our controlled experiments, we ran the latest versions of Android (Ice Cream Sandwich 4.0, and Jelly-bean 4.2) and iOS 6 respectively on our Android and iOS devices. 
We analyze the behavior of OS services and the default applications by first performing a factory reset on these devices, and installing the \platname credentials on this device.
We then test Android and iOS applications by installing the application, interacting with the application for a few minutes, and finally uninstalling the  application. 
During our controlled experiment we use SSL-Bumping to study the behavior of SSL traffic from these applications. 

Our first experiment included manual testing of the top 100 most popular free Android apps from the \emph{Google Play} store and \tbd{} iOS applications from the iOS App store.
For this experiment we first manually installed each application by hand, enter user credentials for accounts like Facebook and Twitter, and toy with the app for \tbd{} minutes. 
In addition to this manual setup, we used an automatic test-click generator to further toy with the Android applications for \tbd{} actions. 
We did not perform this automation step while testing the the iOS applications.
We then uninstall the application and reset the device to test the next application. 

For our second experiment we performed fully-automated tests on 1003 Android applications from a free, third-party Android market.
We perform this test because Android devices can install \emph{Third-party applications} that are not available on the \emph{Google Play} store.
A consequence of this freedom is that numerous third-party app markets are available on the web whose applications have not received research attention.
Our automation used the adb Android command shell to install each app, enable \platname, and start the app.
The system then used Monkey, an adb stress tool, to perform a series of 10,000 actions. 
These actions included random swipes, touches, and text entries.
We then used adb to uninstall the application and reboot the device to forcibly end any lingering connections.

The results of the controlled experiments can be found in \fref{sec:manual-testing}.

\subsection{In The Wild Measurements}

Along with controlled experiments we also conducted a measurement study to characterize the mobile Internet in the wild.
For this study, we deployed two \platname servers, one in USA and one in France, to proxy Internet traffic from 26 devices, 10 iPhones, 4 iPads, 1 iPodTouch, and 11 Android phones.
The Android devices in this dataset include the Nexus, Sony, Samsung, and Gsmart brands while the iPhones include one iPhone~3gs, five iPhone~5, and five iPhone~4S.
These devices belonged to 21 users, volunteers for our IRB approved study.
To protect the identity of the users and their data, we used public key cryptography to encrypt the \emph{pcap} files that log the data traffic flowing through our \platname servers. 
This dataset, called \mobWild, consists of 218 days of data that flowed through our \platname servers; the number days for each user varies from 5 to 215 with a median of 35 days.
We would like to point out that though we performed SSL-Bumping during our controlled experiments, we did not perform SSL-Bumping for the traffic in this dataset.

\begin{table}
\begin{small}
\begin{center}
\begin{tabular}{|p{0.15\columnwidth}|p{0.12\columnwidth}|r|r|r|r|}
\hline
\multirow{2}{*}{\bf IP Protocol} & \multirow{2}{*}{\bf Service} & \multicolumn{2}{|c|}{\bf Android} & \multicolumn{2}{|c|}{\bf iOS} \tabularnewline
\cline{3-6}
           &           &  \textbf{Cell.}  &  \textbf{\wifi}  &  \textbf{Cell.}  &  \textbf{\wifi}  \tabularnewline
\hline
\multirow{3}{*}{TCP}
       &  HTTP  & 35.386 & 68.686 & 52.109 & 75.506 \tabularnewline
\cline{2-6}
       &  SSL   & 61.135 & 27.366 & 46.765 & 18.777 \tabularnewline
\cline{2-6}
       &  other & 2.346  & 3.290  & 0.256  & 1.818 \tabularnewline
\hline
\multirow{2}{*}{UDP}
       &  DNS   & 0.682  & 0.496  & 0.545  & 0.305  \tabularnewline
\cline{2-6}
       &  other & 0.316  & 0.098  & 0.286  & 3.583  \tabularnewline
\hline
 Other &  -     & 0.135  & 0.064 & 0.039  & 0.011  \tabularnewline
\hline
\multicolumn{2}{|c|}{\emph{total}} & 100.00 & 100.00 & 100.00 & 100.00 \tabularnewline
\hline
\end{tabular}
\end{center}
\end{small}
\caption{Traffic volume (in percentage) of popular protocols and services on Android and iOS devices over cellular and \wifi.
\emph{TCP flows are responsible for more than 90\% of traffic volume. Traffic share of SSL over cellular networks is more than twice the traffic share of SSL over \wifi.}} 
\label{tab:summaryIOSAndroidTraffic}
\end{table}

A mobile devices can tunnel the traffic through our \platname servers using either \wifi or cellular networks. 
We estimate the access technology with the description of the AS through which the mobile client connects to our \platname server. 
We get this AS description by performing a \emph{WHOIS} lookup on the IP address used by the mobile client to tunnel Internet traffic. 
For our analysis, we use the WHOIS databases available at \emph{whois.cmyru.com} and \emph{utrace.de}.
We use the information from these \emph{WHOIS} databases to manually classify the ASes to be either cellular or \wifi.
Our dataset consists of data traffic from 54 distinct ASes, of which we classify 9 to be belong to cellular networks.
The devices connected to our system from at most two cellular ASes.
In contrast, a median of 4 \wifi ASes were observed per device and for one device we observed traffic from 25 different \wifi ASes that are spread across 5 countries. 

In \fref{tab:summaryIOSAndroidTraffic} we summarize \mobWild based on the classification performed using Bro~\cite{bro}.
Bro classifies IP flows using the protocol field in the IP header.
We use this classification to label flows as either TCP, UDP, or \emph{other}; flows that are neither TCP nor UDP are classified as \emph{other}. 
Bro further uses the well defined port numbers to identify the services that use TCP.
We use this classification to label flows as either HTTP, SSL (which includes HTTPS, IMAP, etc.) or \emph{other} flows; TCP flows that are not classified as either HTTP or SSL are classified as \emph{other}.
In \fref{tab:summaryIOSAndroidTraffic}, we observe that more than 90\% of the traffic in our dataset is either HTTP or SSL. 
We also observe that the share of HTTP volume over \wifi and cellular are significantly different. 
As detailed in \fref{sec:.}, this difference is primarily due to the use of \wifi to transfer media content.
We also observe the share of SSL traffic over cellular networks is considerably larger compared to \wifi networks.
This increase is a result of the reduced share of media traffic and the use of email and for social networking applications that rely on SSL.
We detail the HTTP and SSL traffic from iOS and Android devices in \fref{sec:}

In summary, we use \platname to perform controlled experiments and in the wild measurements to characterize mobile Internet traffic. 
We now present the results of our experiments and measurements study. 


% \subsubsection{Access Technology Estimation}

% \platname's passive traffic monitoring implies that we need to estimate the access technology, cellular or \wifi, used by the device.
% We estimate the access technology using the description of the AS through which the mobile client connected to our \platname server. 
% We get the AS description by performing a \emph{WHOIS} lookup on the IP address used by the mobile client to tunnel Internet traffic. 
% For our analysis, we use the WHOIS databases available at \emph{whois.cmyru.com} and \emph{utrace.de}.
% We use the information from these \emph{WHOIS} databases to manually classify the ASes to be either cellular or \wifi.
% Our dataset consists of data traffic from \tbd{} distinct ASes, of which we classify \tbd{} as ASes belonging to cellular networks.
% We observe that these \tbd{} ASes belong to \tbd{} distinct ISPs. 
% This observation confirms our intuition that studies based traces from a single ISP~\cite{maier:mobtraffic, vallina-rod:ads}, shall not be able to analyze how specific users use mobile devices. 

% A shortcoming of this approach is when mobile clients are served by a \wifi access point that internally uses a cellular connection to connect the Internet.
% In this case, though the device uses \wifi to connect to the Internet, our servers will observe the connection from a cellular AS. 
%\subsubsection{Overview of Measured Traffic}


%   {\bf iOS Push Notifications.}
%   The applications running on iOS devices receive notifications from Web services using \emph{iOS push notification}. 
%   Push notifications allow an application to alert the user of updates (\eg{}, Facebook messages) while the phone is idle/not in active use.
%   iPhone users are typically warned not to enable push notifications for too many apps due to the potential for these background tasks to (1) consume bandwidth resources and (2) consume battery resources, all without any active user behavior.
%   However, these warnings come with little quantification of exactly {\it how much} an application's push notifications might impact battery life or bandwidth; to date the research community has not measured these properties due to the closed-source nature of iOS and consequent difficulty to measure these properties.
%   Nevertheless, with \platname we can monitor the traffic generated due to push notifications and thus quantify the impact of push notifications despite the iOS lockdown; to the best of our knowledge this is the first measurement characterization of iOS push notifications.

%   Our experiments to study push notifications proceeded as follows:\tbd{...Ashwin?}

%   The results of these tests can be found in \S\ref{sec:pushresults}.

%   {\bf Android Applications.}
%   Both Android and iPhone apps generate traffic to load and upload user data, app content, and advertisements.
%   Although users are informed upon application installation whether or not an app is allowed to access the Internet, the user is unaware {\it what} data is sent, {\it how much} data is sent or accessed, or {\it with whom} the app communications.
%   We define a `well-behaved' application as one which (a) makes limited use of network and battery resources (\ie{} by accessing little bandwidth and by batching traffic to allow radio shutdown during idle periods); (b) contacts only those servers necessary to perform application behavior (\ie{} contacting only a limited number of advertising networks and no tracking sites); and (c) not leaking any personally identifiable information over the network, (\ie{} using HTTPS whenever uploading needed private information like email addresses, and never uploading unnecessary personal information like address book contents or device IMEI). 
% \tbd{Justine: I need your help to rewrite the text here. I have put some crappy text as placeholder.}

%   We test how many applications actually meet these criteria of well-behaved network usage, we performed controlled experiments on blank Android smartphones, iteratively installing, playing with, and monitoring the behavior of hundreds of Android apps whilst running the \platname app in the background.
%   We tested the top 100 most popular free Android apps manually -- installing each app by hand, entering user credentials for accounts like Facebook and Twitter, and toying with the app. In addition to this manual setup, we used an automatic test-click generator to further toy with the app. Afterwards, we uninstalled the app and reset the device.
  
%   Android, unlike iOS, allows users to `side-load' third-party apps on to their device; consequently there are numerous third-party app markets on the web in addition to Google's official Play Store.
%   To study these apps, we performed fully-automated tests on 1003 apps from a free, third-party app market.
%   Our automation used the adb Android command shell to install each app, enable \platname, and start the app.
%   The system then used Monkey, the built-in adb stress tool, to perform a series of 10,000 actions. These actions consisted of random swipes, touches, and text entries.
%   The system then once again used adb to uninstall the app and reboot the device (thus ending all lingering connections and metadata from the previous app.)

%   The results of these tests can be found in \S\ref{sec:androidresults}.
%   {\bf iPhone Applications.}
% \tbd{Dave/Ashwin: Complete the text here for iOS -- subset of apps}. 
% Dummy text for the paragraph.
% \tbd{Dave: Text Here}

% \subsection{In The Wild}

% Along with controlled experiments we also conducted a measurement study to characterize the mobile Internet in the wild.
% We deployed two \platname servers in USA and one server in France. 
% These servers tunnel Internet traffic using VPNs from 25 devices, belonging to 19 users who are volunteers for our IRB approved study.
% To protect the identity of the users and their data, on each server we use public key cryptography to encrypt the files that log the data traffic that flow through the server. 
% We call this dataset the \mobWild dataset. 

% The 25 devices that contribute to the \mobWild dataset consists of 10 iPhones, 4 iPads, 1 iPodTouch, 9 Android phones, and 1 Android tablet.
% Though \emph{tablets} can access the Internet via a cellular data connections, for the \mobWild we included tablets that only use \wifi to access the Internet.
% The Android devices in this dataset include the Nexus, Sony, Samsung, and Gsmart brands.

% This dataset consists of 202 days of data that flowed through our VPN servers; the number days for each user varies from 5 to 198 with a median of 35 days.

% We estimate the access technology used by the mobile device by performing a \emph{WHOIS} lookup on the IP address used by the mobile client for creation of the VPN tunnel. 
% We use the WHOIS databases available at \emph{whois.cmyru.com} and \emph{utrace.de} to get the ISP details.
% We observe that ISPs that provide Internet access over cellular connections use dedicated ASes for cellular traffic. 
% We use the information provided by the \emph{WHOIS} databases to manually classify the ASes used by the mobile devices to be either cellular or \wifi.
% This classification gives incorrect results when mobile clients are served by a \wifi access point that internally uses a cellular connection to connect the Internet.
% In this case, though the device uses \wifi to connect to the Internet, our servers will log the connection to be from a cellular ISP. 

% \tbd{we need some wording and consitency for the usage of ISP -- for example ATT can provide cellular and DSL. Also mobile data cannot be used and we need some word for cellular data and wifi data and this must be defined in the dataset description.}

% Based on the above classification of access technology and ISPs, our dataset consists of data traffic from 52 distinct ISPs, of which 10 provided cellular services.
% Of the 18 devices that used cellular data, we observed that 15 devices restricted their cellular data traffic to one ISP each; we observed that the other three devices accessed the Internet using the services of two different ISPs.
% We observed that the devices in our dataset used a higher number of \wifi ISPs.
% We observed a median of 4 \wifi ISPs per device with a maximum of 25 \wifi ISPs that were used by one device.
% %The device that used \tbdv{24} distinct ISPs was used by a user who traveled across 6 different countries during the data collection process. 
% This observation confirms our intuition that studies based traces from a single ISP~\cite{maier:mobtraffic, vallina-rod:ads}, shall not be able to analyze how specific users use mobile devices. 



\subsection{Ethical Issues}
Capturing all of a subject's Internet traffic raises significant
privacy concerns. Our IRB-approved study entails informed consent from
subjects who are interviewed in lab, where the risks and benefits of
our study are clearly explained. Subjects are incentivized to use the
VPN though a lottery for Amazon.com gift certificates. All data from
tcpdump is encrypted before touching persistent storage; the private
key is maintained on separate secure severs and only approved
researchers can access it. Users may delete their data and/or disable
monitoring at any time. For privacy reasons, we cannot make this data
publicly available.


%%% Local Variables: 
%%% mode: latex
%%% TeX-master: "main"
%%% End: 
