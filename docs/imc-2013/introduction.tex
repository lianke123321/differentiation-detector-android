\section{Introduction}
\label{sec:introduction}

Mobile systems consist of walled gardens inside gated communities, \ie{}, locked-down operating systems running on devices that interact over a closed and opaque mobile network. 
As a result, characterizing Internet traffic naturally generated by mobile devices remains an open problem. 
\tbd{Why do we care? variety of options available, different access technologies, data plans, news OSes, new versions of applications, decreasing quota}

The key challenge is that mobile devices and their OSes provide no built-in service for monitoring and reporting \emph{all network traffic}. 
We strongly believe that a comprehensive network usage analysis must not be limited to specific mobile OSes, access technology, device manufacturer, installed applications, and user behavior. 
Previous works miss out on at least one of the above dimensions of mobile Internet traffic ~\cite{vallina-rod:ads,gerber:passivespeed,chen:wifi,enck:taintdroid,wang:middleboxes,sommers:cellwifi}, thus provides only partial views of network activity -- compromising network coverage. 
In this work, we are the first to present an approach that compromises none of these, potentially enabling a large-scale deployment and comprehensive view of  mobile Internet traffic across carriers, devices, applications versions, and access technologies.

%For example, researchers with privileged access to data gathered from a cellular carrier's network can 
%report on activity for a large group of users; however, this view does not cover WiFi traffic nor 
%does it speak to traffic generated through other cellular networks.~\cite{anyattstudy} Likewise, studies that instrument 
%an enterprise WiFi network can view traffic from devices subscribing to multiple cellular carriers, but not 
%the traffic flowing over cellular links~\cite{someonedidthis}. Last, researchers have used a fleet of devices running a 
%custom version of Android to perform detailed logging of network activity~\cite{mobilelab}. This can capture all of a device's network 
%traffic but is restricted to measuring Android behavior. Further, subjects must root their own devices or to 
%switch from their primary device for the duration of the study -- barriers that can limit deployment 
%size and introduce bias. 
%\tbd{JS: Concerned about all of this related work -- seems like there's a lot! Also -- why is the fact that WiFi vs Celular traffic a big deal? Apps are apps right? Data is data? I guess performance questions come up but isn't okay that the two are characterized separately? Further, meddle introduces big performance changes so...}
%\drc{TODO: make this a less detailed paragraph -- sum it all up in one sentence with three refs, then say we are the first to to avoid the 
%otherwise inherent trade-offs.}

This paper is the first to explore the opportunities for mobile traffic measurement through indirection. 
Specifically, we exploit the fact that most mobile OSes support proxying via virtual private networks (VPNs). 
By sending mobile Internet traffic through a proxy server under our control (an approach we call \platname), we can monitor all flows regardless of device, OS or access technology. 
Importantly, installing a VPN configuration requires neither a new app to be installed nor does it require special or new privileges, thus facilitating large-scale deployment on unmodified device OSes.

We report the results of a 7-month IRB-approved measurement study using this approach both in the lab environment and with human subjects in the wild. 
After demonstrating that our approach incurs reasonable overheads, we describe our measurement methodology and how we use \platname to measure the impact of device OS, apps and service provider on Internet traffic.

Our key contributions are as follows:
\begin{packeditemize}
\item We demonstrate the feasibility of proxy-based measurement for characterizing mobile Internet traffic for iOS and Android. 
\platname captures all Internet traffic generally with less than 10\% power and packet overheads, and negligible additional latency. 
We will make the \platname software and configuration details open source and publicly available by the time of publication.
\item \platname also offers a plugin infrastructure that can be used to inspect and modify the packets in transit. We use this infrastructure to man-in-the-middle SSL connections when we manually tested 209 iOS and 100 Android applications, and performed automated tests on 908 Android applications from a third party store. 
\item We used our experiments to come up with a classification technique to identify the mobile applications and the Web-services responsible for the traffic captured by \platname, and identify personally identifiable leaks by the applications. Our classification technique was able to correctly identify 89.2\% of iOS and 64.1\% of Android applications and Web-services accessed during these experiments. 
\item During our experiments we observed that iOS and Android applications leak the device identifiers and other personally identifiable information such as email address in the clear. Specifically, we observe that one of the most popular iOS application used by US physicians sends the user's first-name, last-name, email, password, and zip-code in the clear. 
\item \platname allows pervasive monitoring, and we used \platname for a 7-month IRB approved study on 26 devices. The key take home is that these measurements were did not require custom OSes, ISP support, or support from marketplaces, warranty voiding of devices. 
\item During our measurement study, we observe that most traffic is either compressed, or encrypted, thus limiting the opportunities for additional traffic-volume optimization. Furthermore, we observe that popular ads and analytic sites were tracking our devices.
\end{packeditemize}
% \item A measurement study of app behavior (both popular and otherwise) from Android and iOS. We observe \tbd{values come here}. 
% \tbd{say something about how we can directly observe differences in the network behavior of identical apps designed for different OSes.}
% \item An analysis of privacy leaks in the mobile environment. \tbd{Results based on Amy work}.
% %\eat{\item A new measurement technique for detecting ISP interference with arbitrary Web site content.}
% \item \tbd{Results from an on going IRB based study of 30 users. We use these results to compare our observations from exisiting studies. The key take home is that these measurements were did not require custom OSes, ISP support, or support from marketplaces, warranty voiding of devices.}
% \end{packeditemize}
% \eat{
% \tbd{The above is a laundry list -- can we highlight three or four things at most? Sort of macro points and get to the details later?}
% The remainder of the paper is organized as follows:
% \tbd{Things to highlight in Intro\\
% Tools\\
% Techniques\\
% Methodology\\
% Insights}
% \tbd{Justine: Primary concern is that the secondary paragraph doesn't sell this as very novel -- others have all done this before is sort of the lesson I learned there. What's new?
% After reading this, I think we need to say, "comprehensive network usage analysis" is part of what's new here - we can track users across multiple networks and platforms; this allows us to say that x fraction of traffic is over 3G and y fraction is over WiFi, that bandwidth usage changes by x percent wen moving betwen 3G and Wifi." Because it's easy to install, this means that we can study large numbers of people (given IRB constraints) with little overhead.
% One additional thing is we should call out what findings we have are new -- it doesn't matter if our methodology is new at all if we have sexy new discovery X property of network traffic/app behavior/etc.
% }}

The rest of the paper is organized as follows. 
We present an overview of platname and argue its feasibility in \fref{sec:platform}, followed by the details of our experiments and measurement dataset is presented in \fref{sec:dataset}.
In  \fref{sec:classification-methodology} we present our technique to classify the mobile applications and discuss the privacy invasive nature of these applications in \fref{sec:characterize-app}. 
We discuss the related work in \fref{sec:related} and our conclusions in \fref{sec:conclusion}.

%%% Local Variables: ***
%%% mode:latex ***
%%% TeX-master: "main.tex"  ***
%%% End: ***


