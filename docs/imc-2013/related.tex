\section{Related Work}
\label{sec:related}

The network behavior of mobile systems has implications for battery life, 
data-plan consumption, privacy, security and performance, among others. 
When attempting to characterize this behavior, researchers face a number 
of trade-offs: compromising network coverage (limiting the number and type of ISPs measured), 
portability (limiting the device OSes) and/or deployability (limiting subscriber coverage).
\platname compromises 
none of these, enabling comprehensive measurements across carriers, devices and access 
technologies. Table~\ref{tab:relatedCompare} puts our approach in context with previous 
approaches for measuring the network behavior of mobile systems. 

\begin{table*}[t]
\begin{center}
{\footnotesize
\begin{tabular}{|l|l|l|l|l|}
\hline
 & \textbf{Network Coverage} &  \textbf{Portability} &  \textbf{Deployment model} &   \textbf{Meas. Type}  \\ \hline
AT\&T/Telefonica study~\cite{vallina-rod:ads,gerber:passivespeed} & Single carrier & All OSes & Instrument cell infrastructure & Passive \\ \hline
WiFi study~\cite{chen:wifi} & Single WiFi network & All OSes & Instrument WiFi network & Passive \\ \hline
PhoneLab/TaintDroid~\cite{enck:taintdroid} & Multiple networks & Android & Install custom OS & Active/Passive \\ \hline
MobiPerf~\cite{wang:middleboxes}/SpeedTest~\cite{sommers:cellwifi} & Multiple networks & Android & Install App & Active \\ \hline
\platname & Any network & Android / iOS & VPN configuration & Passive \\ \hline
\end{tabular} }
\end{center}
\vspace{\postfigspace}
\caption{Comparison of alternative measurement approaches. \platname is the first approach to cover all access networks and most device OSes, capturing 
network traffic passively and with low overhead via VPN proxying.}
\label{tab:relatedCompare}
\vspace{\postfigspace}
\end{table*}%

Traces from mobile devices can inform a number of interesting analyses. Previous work 
uses custom OSes to investigate how devices waste energy~\cite{pathak:eprof}, network bandwidth and 
leak private information~\cite{enck:taintdroid,hornyack:appfence}. Similarly, AppInsight~\cite{ravindranath:appinsight} and PiOS~\cite{egele:pios} can inform 
app performance through binary instrumentation and/or static analysis. In this work, we explore the opportunity to use network traces 
alone to reveal these cases without requiring any OS or app modifications.

Network traces from inside carrier networks provide a detailed view for large numbers 
of subscribers. For example, Vallina-Rodriguez~\etal~\cite{vallina-rod:ads} uses this approach to characterize performance and 
the impact of advertising. Gerber \etal~\cite{gerber:passivespeed} similarly use this approach to 
estimate network performance for mobile devices.  \cite{maier:mobtraffic} \cite{chen:wifi}
Similar to these approaches, \platname provides continuous passive monitoring of mobile network 
traffic; however, \platname is the first to do so across all networks to which a device connects.

Last, active measurements~\cite{wang:middleboxes,sommers:cellwifi} allow researchers to understand network topologies and instantaneous 
performance at the cost of additional, synthetic traffic for probing. In contrast, \platname uses 
passive measurements to characterize the traffic that devices
naturally generate.

%%% Local Variables: 
%%% mode: latex
%%% TeX-master: "main.tex"
%%% End: 
