\section{Motivation}
\label{sec:motivation}

The current mobile ecosystem is very opaque and it offers researchers  a
limited view into how mobile devices and the installed apps generate
network traffic. This lack of transparency is currently being addressed by
either instrumenting the mobile operating system (OS), instrumenting the
binaries, static analysis of app binaries, or relying on ISP traces. In
this section, we discuss each of these techniques and use their 
shortcomings to motivate the need for a new platform for mobile
measurements.      

% Why it is done. Papers doing it. Short comings. 
Instrumenting a mobile OS system provides researchers a fine grained view
of the apps and OS in action. Taintdroid~\cite{enck:taintdroid}, one of the
seminal works of instrumenting the Android OS, and its extensions have
therefore been widely used to study Android apps in action. For example,
Hornyack~\etal~cite{hornyack:appfence} rely on Taintdroid to detail the
internals of the 1100 most popular Android apps. The major shortcoming of
instrumenting OSes is that the measurement results are limited to a 
specific OS version and apps written for that OS version. This implies that
instrumenting OSes cannot be used for longitudinal studies that need to
consider changes in not only the app code but also the OS code. 
Furthermore, because the app code is tightly coupled to the API provided by
the underlying OS, the results obtained by instrumenting  one mobile OS
cannot be extrapolated to other mobile OSes. Instrumenting an OS also
results in a high barrier to entry for studies that require participation
of end users  who may be unwilling to modify the underlying OS. 

Instrumenting app binaries at predefined code points can be used to detail
the behavior of a specific set of apps. One of the biggest advantages of 
instrumenting apps is its low barrier to entry because it does not require
an OS modifications. This low barrier to entry was one the key motivations
for the development of AppInsight~\cite{ravindranath:appinsight}. Indeed
AppInsight can provide developers improve the app quality, however, its
scope is limited to the apps that have been instrumented, the marketplaces
from where the apps are downloaded, and the OS version for which the
instrumented app was written. Furthermore, each new version of the app
needs to be instrumented.

Static analysis of the app code is used to study apps when the apps and the
underlying OS are secured to avoid being tampered. For example,
Egele~\etal~\cite{egele:pios}, perform static analysis of 1400 IOS apps by
static analysis. The authors observe that more than half of the apps they
analyzed leak the unique ID of the device. 




{Identifying Diverse Usage Behaviors of Smartphone Apps} {IMC 2012} 
        Gather traces from ISP measurements: IMC 2012 paper on Ads.
Requires
support from ISPs which includes contracts limiting the quality and
quantity of
data. Data is non-comprehensive because users can use different service
providers for Wi-F and Cellular.
    In summary these techniques have a high barrier to entry and despite
this
high barrier they come with serious limitations. 

%%% Local Variables: 
%%% mode: latex
%%% TeX-master: "main"
%%% End: 





