%Note one sentence in one text line.
\section{Motivation}
\label{sec:motivation}

The current mobile ecosystem is very opaque and it offers researchers a limited view into how mobile devices and the installed apps generate network traffic. 
This lack of transparency is currently being addressed by either instrumenting the mobile operating system (OS), instrumenting the binaries, static analysis of app binaries, or relying on ISP traces. 
In this section, we discuss each of these techniques and use their  shortcomings to motivate the need for \platname, a VPN based platform to monitor mobile Internet traffic.      

Instrumenting a mobile OS system using tools such as Taintdroid~\cite{enck:taintdroid} and AppFence~\cite{hornyack:appfence} provides researchers a fine grained view of the apps and OS in action. 
For example, Hornyack~\etal~\cite{hornyack:appfence} use AppFence to detail the internals of the 1100 most popular Android apps. 
The major shortcomings of instrumenting OSes is that it can result in warranty voiding of the device and that the measurement results are limited to a specific OS version and apps written for that OS version.
Longitudinal studies that detail the impact of OS code changes and app code changes cannot be performed by instrumenting OSes.
Furthermore, because the app code is tightly coupled to the API provided by the underlying OS, the results obtained by instrumenting  one mobile OS cannot be extrapolated to other mobile OSes. 
Instrumenting an OS also results in a high barrier to entry for practical studies that require participation of end users  who may be unwilling to modify the underlying OS and void the warranty of their devices. 
\tbd{The tone should move towards network monitoring and that OS instrumentation is an overkill for network measurements.}

Instrumenting app binaries at predefined code points can be used to detail the behavior of a specific set of apps. 
One of the biggest advantages of  instrumenting apps is its low barrier to entry because it does not require an OS modifications. 
This low barrier to entry was one the key motivations for the development of AppInsight~\cite{ravindranath:appinsight}.
Indeed AppInsight can provide a detail analysis of apps, however, in terms of the network footprint of the app, the scope of AppInsight is limited to the instrumented apps, the marketplaces from where the apps are downloaded, and the OS version for which the app was instrumented.
Furthermore, each new version of the app needs to be instrumented.

Static analysis of the app code is used to study apps when the apps and the underlying OS are secured to avoid being tampered. 
For example, PiOS~\cite{egele:pios} was used to perform static analysis of 1400 IOS apps by static analysis. 
The authors of PiOS observe that the unique ID of the device is leaked by more than half of the apps they analyzed. 
A shortcoming of this study is that the PiOS can access the app binaries only after the iOS device is jail-broken, thus voiding the warranty of the device.
Furthermore, like AppInsight~\cite{ravindranath:appinsight}, the results of PiOS are limited to the iOS operating system. \tbd{Text for SPARTA project at UW.}

ISP traces are useful to study mobile devices in the wild. 
Viallina-Rodriguez~\etal~\cite{vallina-rod:ads} use an ISP trace of 3 million subscribers to detail the impact of ads and analytics on the mobile data and energy consumption. 
Similarly, Maier~\etal~\cite{maier:mobtraffic} study the mobile traffic by looking at the DSL traces from a popular European ISP. 
However, these studies cannot provide a comprehensive view of the traffic from mobile devices because users can access the Internet using different ISPs depending on their location and the access technology used to connect to the Internet. 
For example, the home \wifi and office \wifi may be served from ISPs that are different from the ISP used for cellular data traffic. 

In summary, existing solutions to measure the network characteristics of mobile network traffic fall short of being either portable, pervasive, passive, or deployable. 
In the next section we show that a VPN based platform can address these issues to get a network perspective of mobile devices.  

\tbd{Discuss new OSes coming out. .. Ubuntu, Firefox OS, etc}.
