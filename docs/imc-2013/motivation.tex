%Note one sentence in one text line.
\section{Motivation}
\label{sec:motivation}

The current mobile ecosystem is very opaque and it offers researchers  a limited view into how mobile devices and the installed apps generate network traffic. 
This lack of transparency is currently being addressed by either instrumenting the mobile operating system (OS), instrumenting the binaries, static analysis of app binaries, or relying on ISP traces. 
In this section, we discuss each of these techniques and use their  shortcomings to motivate the need \platname, a VPN based platform for mobile measurements.      

% Why it is done. Papers doing it. Short comings. 
Instrumenting a mobile OS system provides researchers a fine grained view of the apps and OS in action. Taintdroid~\cite{enck:taintdroid}, one of the seminal works of instrumenting the Android OS, and its extensions have therefore been widely used to study Android apps in action. 
For example, Hornyack~\etal~\cite{hornyack:appfence} rely on Taintdroid to detail the internals of the 1100 most popular Android apps. 
The major shortcoming of instrumenting OSes is that the measurement results are limited to a specific OS version and apps written for that OS version. 
This implies that instrumenting OSes cannot be used for longitudinal studies that need to consider changes in not only the app code but also the OS code.
Furthermore, because the app code is tightly coupled to the API provided by the underlying OS, the results obtained by instrumenting  one mobile OS cannot be extrapolated to other mobile OSes. 
Instrumenting an OS also results in a high barrier to entry for studies that require participation of end users  who may be unwilling to modify the underlying OS. 

Instrumenting app binaries at predefined code points can be used to detail the behavior of a specific set of apps. 
One of the biggest advantages of  instrumenting apps is its low barrier to entry because it does not require an OS modifications. 
This low barrier to entry was one the key motivations for the development of AppInsight~\cite{ravindranath:appinsight}.
Indeed AppInsight can provide developers improve the app quality, however, its scope is limited to the apps that have been instrumented, the marketplaces from where the apps are downloaded, and the OS version for which the instrumented app was written. 
Furthermore, each new version of the app needs to be instrumented.

Static analysis of the app code is used to study apps when the apps and the underlying OS are secured to avoid being tampered. 
For example, PiOS~\cite{egele:pios} was used to perform static analysis of 1400 IOS apps by static analysis. 
The authors of PiOS observe that the unique ID of the device is leaked by more than half of the apps they analyzed. 
A shortcoming of this study is that the PiOS can access the app binaries only after the iOS device is jail-broken, thus voiding the warranty of the device.
Furthermore, like AppInsight~\cite{ravindranath:appinsight}, the results of PiOS are limited to the iOS operating system. \tbd{Text for SPARTA project at UW.}

ISP traces are useful to study mobile devices in the wild. 
Viallina-Rodriguez~\etal~\cite{vallina-rod:ads} use an ISP trace of 3 million subscribers to detail the impact of ads and analytics on the mobile data and energy consumption. 
Similarly, Maier~\etal~\cite{maier:mobtraffic} study the mobile traffic by looking at the DSL traces from a popular European ISP. 
However, these studies cannot provide a comprehensive view of the traffic from mobile devices because users can access the Internet using different ISPs depending on their location and the access technology used to connect to the Internet. 
For example, the home wi-fi and office wi-fi may be served from ISPs that are different from the ISP used for cellular
data traffic. 

\tbd{In summary, portability, pervasiveness, ubiquitous, and deployable missing in existing platforms. 
In the next section we show that a VPN based platform can address these issues to get a network perspective of mobile devices}. 
 





