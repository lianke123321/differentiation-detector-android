\section{Application Characterization}
\label{sec:characterize-app}

To analyze applications, we need to first identify the applications that is responsible for the flows. 
Once identified we need to analyze the applications. 
For analysis we concentrate on including {\it how much} data is sent or accessed, {\it what} data is sent,  or {\it with whom} the app communications.
``How much'' is important to conserve both bandwidth caps and battery capacity.
``With whom'' is important to protect users from excessive tracking.
Finally, ``what data'' is important because apps may unnecessarily leak personally identifiable information (PII) such as user email address, IMEI, contact information, or other stored data either to the app provider or worse, to any eavesdropper on a public WiFi connection.

\subsection{Application Usage}


\subsubsection{Bandwidth Usage with time of specific application}

X-axis is the user and y-axis is the share over cell, wifi, and combined for each user.

X-axis is the day and y axis is the share over cell and wi-fi.

SSL + Cellular 

\subsubsection{Multimedia Traffic in he Wild}

X-axis is the user and y-axis is the share over cell, wifi, and combined.

\subsection{User Tracking}

\subsubsection{Behavior of Ads and Analytics}

Fraction of ads and analytics over cell and wifi. 
The number of devices that were tracked by ads and analytics sites.

\subsubsection{Personally Identifiable Information}

The frequency of tracking events per user per day in http.

SSL Bumping controlled experiments: sites that leaked pii in the clear and in ssl.

CDNs and other sites through which the traffic flowed.

\subsection{Discussion}



%%% Local Variables: 
%%% mode: latex
%%% TeX-master: "main"
%%% End: 



% \begin{table}
% \begin{tabular}{|l|l|}
% \hline
% \useragent Prefix& OS \tabularnewline
% \hline
% AppleCoreMedia/1.0 & iOS \tabularnewline
% stagefright/1.2 & Android \tabularnewline
% Dalvik/1.6 & Android \tabularnewline
% Linux; Android & Android \tabularnewline
% com.google.android.youtube & Android \tabularnewline
% \hline
% \end{tabular}
% \caption{Prefix of \useragent string while streaming youtube videos. \emph{While iOS devices use AppleCoreMedia for more than 95\% of YouTube traffic, Android devices use a variety of \useragent strings depending on the Android version, YouTube application version, and the version of various applications from which YouTube videos are viewed.}}
% \label{tab:top-user-agents}
% \end{table}





