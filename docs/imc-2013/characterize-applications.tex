\section{Privacy Invasive Services}
\label{sec:characterize-app}
\tbd{to revise}
We now use pervasive nature of \platname to detail  the privacy invasiveness of applications and Web-services. 
For our analysis we concentrated on the data sent from the mobile devices with a focus on the {\it what} data is sent,  {\it to whom} is the data sent, and {\it how frequently} is data sent.
%To answer these questions, we rely on the controlled experiments and the \mobWild dataset.

\begin{table*}[t]    
    \centering
    \begin{small}
    \begin{tabular}{|l|l|l|l|l|l|l|l|l|l|}
       \hline
       {\bf Store}&{\bf Platform}&{\bf \# Apps}&{\bf Email}& {\bf Location}& {\bf Name} &{\bf Password}& {\bf Device ID}& {\bf Contacts}& {\bf IMEI}\\
       \hline
       App Store&iPhone&209&13 (6.2\%) &20 (9.5\%)&4 (1.9\%)&0 (0\%)&16 (7.6\%)&0 (0\%)&0 (0\%)\\
       \hline
       Google Play&Android&100&3 (3\%)&10 (10\%)&2 (2\%)&1 (1\%)&21 (21\%)&0 (0\%)&13 (13\%)\\
       \hline
       Third Party&Android&908&1 (0.1\%)&32 (3.5\%)&2 (0.2\%)&0 (0\%)&95 (10.4\%)&4 (0.4\%)&48 (5.3\%)\\
       \hline
    \end{tabular}
    \end{small}
    \caption{Summary of personally identifiable information leaked in plaintext (HTTP) by Android and iPhone applications. \emph{The popular iOS applications tend to leak the location information in the clear while Android applications leak the IMEI number and Android ID in the clear.}}
    \label{tab:pii}
\end{table*}

\subsection{Personally Identifiable Information Leaks}

For our experiments, we created fake user accounts with fake contact
information, and fake Twitter and Facebook accounts.  Our goal is to
detect if any Personally Identifiable Information (PII)---email
address, phone number, IMEI number---stored on the device is leaked
across the network over HTTP or HTTPS (using the SSL bumping plugin).
While we acknowledge that some of this information may be relevant for
the application, we strongly believe that this information should
never travel across the network in plaintext (HTTP).


In \fref{tab:pii}, we present the different personally identifiable
information leaked for both Android and iPhone apps.  We observe that
the IMEI, a unique identifier tied to a phone, is the most commonly
leaked PII for Android applications.  This IMEI can be used to track
and correlate a user's behavior across Web-services.  Similarly, we
observe that Android applications leak the Android ID, a unique
identifier tied to an Android device.  In \fref{tab:pii}, we also
observe that other information like contacts, emails, and passwords
are leaked in the clear.  The email address, the address used to sign
up for the services, was leaked in the clear by 13 iOS and 3 Android
applications from our set of popular applications.

\begin{table}
    \centering
    \begin{small}
    \begin{tabular}{|l|c|c||c|}
       \hline
       {\bf Host}&{\bf IMEI}&{\bf Device ID} & {\em Ads \& Analytics} \tabularnewline
       \hline              
       chartboost.com                & \checkmark & \checkmark & \checkmark  \tabularnewline
       tapjoyads.com                 & \checkmark & -          & \checkmark  \tabularnewline
       getjar.com                    & \checkmark & \checkmark & -   \tabularnewline
       pocketchange.com              & \checkmark & \checkmark & -   \tabularnewline
       iheart.com                    & \checkmark & \checkmark & -   \tabularnewline
       aarki.net                     & \checkmark & -          & \checkmark  \tabularnewline
       zynga.com                     & \checkmark & -          & -   \tabularnewline
       droidsecurity.appspot.com     & \checkmark & -          & -   \tabularnewline
       google.com                    & -          & \checkmark & -   \tabularnewline
       flurry.com                    & -          & \checkmark & \checkmark  \tabularnewline
       groupon.com                   & -          & \checkmark & -   \tabularnewline
       \hline
    \end{tabular}
    \end{small}
    \caption{Top 10 hosts that receive the IMEI or Device ID over HTTPS. \emph{Hosts are ordered by the number of flows that send the IMEI number, followed by the number of flows that send the device ID over HTTPS. Four of the top 10 hosts that receive this information are ads and analytics sites.}}
    \label{tab:pii-leakage-https-sites}
\end{table}

During our experiments, we observed that personally identifiable
information information is also sent over HTTPS.  In the following, we
focus on device identifiers such as the IMEI and the Android device
ID.  In \fref{tab:pii-leakage-https-sites}, we present the top 10
sites ordered by the number of flows that sent the IMEI over HTTPS.  We
observe that four of the top 10 sites that receive this information
are ads and analytics sites.

Finding ads and analytics sites that receive personally identifiable
information from apps is an abuse of the permission the user is giving
to the apps. Indeed, if an app ask to access device identifiers and
the user grant this access, the user is never notified that third
parties will have access to these identifiers. Out of the 77 sites
that received either the IMEI or Device ID in the clear or over HTTPS,
35 sites were third party ads and analytics sites.

In summary, we use our controlled experiments to identify PII leaks on
both HTTP and HTTPS, and we show that PII are leaked to third party
sites such as ads and analytics. These controlled experiments are a
practical use case of \platname, experiments requiring warranty
voiding the devices otherwise. In particular, \platname{} enable to
reveal PII leaks over HTTPS.
%However, we would like to point out that we were not able to analyze traffic in which the data sent was encoded and exchanged as binary objects. 

%Stats value 13 of 22 in the clear device ID, 16 of 39 IMEI clear,  3 of 6 IMEI HTTPs, 3 of 10 device HTTPs}

\subsection{PII in the Wild}
In the previous section, we focused on controlled experiments. We now
focus on the analysis of the \mobWild{} dataset. For this dataset, for
evident ethical reasons, we did not run the SSL bumping plugin because
it would have revealed credentials of most of the applications used by
our pool of real users. Instead, we focus on which information is
leaked in the clear.

\begin{figure} 
\includegraphics[width=\columnwidth]{figures/wordcloud_useragentsignature_location_image.png}
\caption{Applications that send the location information in the clear. \emph{The font size represent the number of flows that sent the location information in the clear.}}
\label{fig:location-wordcloud}
\end{figure}

In \fref{fig:location-wordcloud}, we present a \emph{word cloud} of
the applications that send the location information of the devices.
We observe that a bus service application (\emph{One Bus Away}), the
application that manages the iOS homescreen (\emph{springboard}), and
weather applications (\emph{twc}, \emph{weather}, \emph{hurricane})
were responsible for more 78\% of the flows that sent the location
information in the clear.  Moreover, we observe that 4\% of the flows
sent the location information to ads and analytics sites; more than
80\% of \emph{ad-flows} leaking location information did not include
an application signature in the user-agent field, the rest of the
flows being from apps including browsers, the Facebook app, and angry birds.

In addition, we also observe that the device ID and IMEI
number are leaked in the clear in the \mobWild dataset.  Based on our
classification methodology, we observe that the IMEI number and device
ID is leaked by the Web-browser; we do not observe
any application signature in non-browser flows in the \mobWild dataset
that leaked the IMEI number or the device ID in the clear.  As in the
case of controlled experiments, ads and analytics sites are the most
popular destination for the IMEI number leaks.  Among the 16 sites
that sent the IMEI number in the clear, 10 sites are ads and
analytics sites; the rest of the sites includes sites for games, news,
and manufacturer updates.

%Similarly, from the device of one of the authors of the paper, we observed that the latest three versions YahooMail application, up to the time of the measurements, leaked the user's email address in the clear.




%\tbd{This should come before 
%We first identify A\&A flows using the publicly available database of~\cite{YoyoAds}; we augment this list of domains using recent research on mobile ads~\cite{hornyack:appfence, leontiadis:mobileads}.
%Based on this classification, we observe that the ads and analytics traffic was responsible for up to 6\% of the traffic by volume per device, an observation in line to the one made by Vallina-Rodriguez~\etal~\cite{vallina-rod:ads}}.

%\subsection{Ads and Analytics in the Wild}

\tbd{focusing on ads and analytics is just what you did in the
  previous paragraph. What is new or different in this one.}
We now focus our attention on the extent to which devices in the
\mobWild dataset contact ads and analytics (A\&A) sites, an activity
that is receiving considerable
attention~\cite{roesner:webtrackers,leontiadis:mobileads,vallina-rod:ads}.
Using our classification based on the \httphost, we observe that the
ads and analytics traffic was responsible for up to 6\% of the traffic
by volume per device, an observation in line with the one made by
Vallina-Rodriguez~\etal~\cite{vallina-rod:ads}.  Rather that focusing
on the traffic volume we focus on the extent to which these sites are
able to track the users in the dataset and the applications that
facilitate this tracking.

\begin{table}
\centering
\begin{small}
\begin{tabular}{|p{0.35\columnwidth}|p{0.1\columnwidth}|p{0.15\columnwidth}|p{0.1\columnwidth}|}
\hline
\multirow{2}{*}{\bf Tracker} & \multicolumn{3}{c|}{\bf Number of devices tracked}\tabularnewline
\cline{2-4}
                      &  {\bf Total} & {\bf iOS} & {\bf Android} \tabularnewline
\hline
doubleclick.net       & 26 {\em(all)} & 15 {\em(all)} & 11 {\em(all)} \tabularnewline
\hline
google-analytics.com  & 26 {\em(all)} & 15 {\em(all)}  & 11 {\em(all)} \tabularnewline
\hline
googlesyndication.com & 22 & 12 & 10 \tabularnewline
\hline
admob.com             & 21 & 11 & 10 \tabularnewline
\hline
scorecardresearch.com &  21 & 11 & 10 \tabularnewline
\hline
\end{tabular}
\end{small}
\caption{The top 5 ads and analytics sites that were contacted by the devices in our dataset.
\emph{The sites, doubleclick.net and google-analytics.com, were contacted by all the 26 devices in} \mobWild.}
\label{tab:top-trackers}
\end{table}

In \fref{tab:top-trackers} we present the number of A\&A sites ordered
according to the number of devices that contacted them.  We observe
that all the devices in the \mobWild dataset contacted
doubleclick.com, an ad site, and google-analytics.com, a tracking
site.  Furthermore, we observe that 66.12\% of the volume of ad
traffic in the \mobWild dataset was from the browsers, 6.46\% of the
traffic contained a blank user-agent field, and 4.8\% of the traffic
contained a signature of \emph{Google-Analytics}\footnote{This
  signature was observed even in the flows for users that did not have
  the Google Analytics application installed on the device.}.  The
rest of the traffic contained signatures of other applications such as
Facebook, Pandora, and YouTube.
%\tbd{This must come before Furthermore, we observed that 7\% of the traffic contained the Google-Analytics in the signature field; this signature was observed even in the flows for users that did not have the Google Analytics application installed on the device. Mention that we will wrongly classify the flows from Google Analytics app as ads and analytics traffic} 




%%% Local Variables: 
%%% mode: latex
%%% TeX-master: "main"
%%% End: 

%       getjar.com     & - S & - S \tabularnewline
%       aarki.net      & - S & - - \tabularnewline
%       chartboost.com & - S & - S \tabularnewline
%       *pocketexchange&   S &   S \tabularnewline
%       *vserv.mobi    & - - & C   \tabularnewline
%       *groupon.com   &     &   S
%       *flurry.com    &     &   S
%       *bankofamerica &     &   S
%       *google.com    & - - & - S

\section{MISC}

% \begin{figure}
% \centering
% \includegraphics[width=\columnwidth]{plots/ads_wild_gatracking.pdf}
% \caption{Number of devices tracked by Google-Analytics. \emph{We observe that devices }}
% \label{fig:tracking-analytics}
% \end{figure}

% \begin{figure}
% \centering
% \includegraphics[width=\columnwidth]{plots/ads_wild_usertracking.pdf}
% \caption{Number of sites that track a user. \emph{We observe that devices }}
% \label{fig:tracking-analytics}
% \end{figure}

% \begin{figure}
% \centering
% \includegraphics[width=\columnwidth]{plots/ads_wild_sitescontacted.pdf}
% \caption{Number of visits to A\&A sites per device. \emph{The error bars indicate the 25$^{th}$ and 75$^{th}$ percentiles. Each visit is a potential tracking visit.}}
% \label{fig:tracking-analytics}
% \end{figure}

% \begin{table}    
%     \centering
%     \begin{small}
%     \begin{tabular}{|l|c|c|}
%        \hline
%        {\bf Host}&{\bf IMEI}&{\bf Device ID}\tabularnewline
%        \hline              
%        tapjoyads.com  & Y & - \tabularnewline
%        zynga.com      & Y & - \tabularnewline
%        iheart.com     & Y & Y \tabularnewline
%        google.com     & - & Y \tabularnewline
%        flurry.com     & - & Y \tabularnewline
%        \hline
%     \end{tabular}
%     \end{small}
%     \caption{Hosts to which the IMEI or Device ID was sent in the clear and over HTTPS. \emph{The value Y for a column implies that data was sent over HTTP and HTTPS. We order these hosts based on the number of flows that sent the IMEI number in the clear and over HTTPS.}}
%     \label{tab:pii-leakage-sites}
% \end{table}
