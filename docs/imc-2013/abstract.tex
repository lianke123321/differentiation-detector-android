%One sentence in every text line.
\begin{abstract}
Characterizing Internet traffic naturally generated by mobile devices remains an open problem. The key challenge is that mobile devices and their OSes provide no built-in service for monitoring and reporting all network traffic. The result is that researchers are left with partial views of network activity-- through monitoring inside mobile carrier networks, from WiFi access points or logging data on custom OSes.

In this paper, we take an alternative approach: measurement through indirection. Specifically, we exploit the fact that most mobile OSes support proxying via virtual private networks (VPNs). By sending mobile Internet traffic through a proxy server under our control, we can monitor all flows regardless of device, OS or access technology. Further, our solution is amenable to large-scale deployment because it requires no special privileges and can be configured via software on users' existing phones.

We report the results of a 6-month IRB-approved measurement study using this approach both in the lab environment and with human subjects in the wild. After demonstrating that our approach incurs reasonable overheads, we describe our measurement methodology and how we use \platname to measure the impact of device OS, apps and service provider on Internet traffic.

\tbd{Monitoring?-- We also perform controlled experiments}
\end{abstract}

% 1) Motivation/problem statement: Why do we care about the problem? What practical, scientific, theoretical or artistic gap is your research filling?
% 2) Methods/procedure/approach: What did you actually do to get your results? (e.g. analyzed 3 novels, completed a series of 5 oil paintings, interviewed 17 students)
%3) Results/findings/product: As a result of completing the above procedure, what did you learn/invent/create?
%4) Conclusion/implications: 

%%% Local Variables: ***
%%% mode:latex ***
%%% TeX-master: "main.tex"  ***
%%% End: ***
