%One sentence in every text line.
\begin{abstract}
  Characterizing Internet traffic naturally generated by mobile
  devices is an open problem because mobile devices and their
  OSes provide no built-in support to monitor network traffic.
  Therefore, researchers exploring the mobile traffic in the Internet
  either work on real network traces, but without the possibility to
  control the experiment, or on custom OSes requiring to root or
  jailbreak the devices, but with the difficulty to scale the
  experiment to a large number of various devices. 

% This
%   opacity offers researchers with a limited perspective on mobile
%   Internet traffic obtained by either relying on traces from carrier
%   networks, monitoring traffic flowing through \wifi access points, or
%   logging data traffic on custom OSes.

  In this paper, we take an alternative approach: monitoring through
  indirection. Specifically, we exploit the fact that most mobile OSes
  support proxying via virtual private networks (VPNs).  Sending
  mobile Internet traffic through a proxy server under our control
  enables us to monitor all flows regardless of device, OS, or access
  technology.  We argue that our solution, \platname, has reasonable
  overheads and can be configured on existing phones without any OS
  modification. This makes \platname{} feasible for a large variety of
  experiments from a small scale controlled experiment to a
  large-scale experiment with a large variety of devices, OSes, and
  cellular providers.

  We present the architecture of \platname, a software package running
  on a single machine that can monitor all mobile traffic, and that
  provides a convenient plugin infrastructure to analyze and modify
  the mobile traffic on-the-fly. In particular, we present a SSL
  bumping module that can decrypt and uncover most of the SSL
  traffic. Then, using \platname{} on both controlled experiments and
  a 7-month IRB-approved in-the-wild study with a small set of real
  users, we analyze key characteristics of iOS and we compare them
  with Android, such as the push notification services or the
  applications network footprint.

  % We report the results of a 6-month IRB-approved measurement study in
  % which we used \platname to detail the characteristics of OS services
  % and applications.  We also report on controlled experiments
  % performed using \platname not only to detail the behavior of popular
  % and suspicious mobile applications but also to peek into SSL
  % traffic.
\end{abstract}

% 1) Motivation/problem statement: Why do we care about the problem? What practical, scientific, theoretical or artistic gap is your research filling?
% 2) Methods/procedure/approach: What did you actually do to get your results? (e.g. analyzed 3 novels, completed a series of 5 oil paintings, interviewed 17 students)
%3) Results/findings/product: As a result of completing the above procedure, what did you learn/invent/create?
%4) Conclusion/implications: 

%%% Local Variables: ***
%%% mode:latex ***
%%% TeX-master: "main.tex"  ***
%%% End: ***
